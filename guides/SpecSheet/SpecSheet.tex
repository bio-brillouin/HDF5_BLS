\documentclass{article}
\usepackage{enumitem}

\title{Spectrometer Data Storage Specification Sheet}
\author{}
\date{}

\begin{document}

\maketitle

\section{Simplicity}
\subsection{Conceptually Simple}
\begin{itemize}
    \item \textbf{Single measures:} Clearly define how to build a file with a single measure.
    \item \textbf{Attributes:} Clearly define how to store parameters associated with a single measure.
    \item \textbf{Hyper parameters:} Clearly define how to store measures that depend on multiple parameters.
\end{itemize}

\subsection{Practically Simple}
\begin{itemize}
    \item The user should have a GUI that builds the file for him unambiguously.
\end{itemize}

\section{Universality}
\subsection{Techniques}
\begin{itemize}
    \item We should be able to store any type of spectra from different techniques.
\end{itemize}

\subsection{Dimensionality}
\begin{itemize}
    \item We should be able to store spectra without any dimensionality limitation.
\end{itemize}

\section{Unify Treatment}
\subsection{Obtention of a Custom PSD}
\begin{itemize}
    \item The format should allow any steps leading to the obtention of a doublet \{PSD, Frequency\}.
\end{itemize}

\subsection{Unified Treatment}
\begin{itemize}
    \item Once a \{PSD, Frequency\} doublet is obtained, the format should allow to treat unambiguously the data from the GUI.
\end{itemize}

\subsection{Multiple Treatments}
\begin{itemize}
    \item The format should allow the user to store different treatments.
\end{itemize}

\subsection{Process}
\begin{itemize}
    \item The format should allow the user to store the process used for the treatment.
\end{itemize}

\section{Expandability for Future Applications}
\begin{itemize}
    \item The file format should allow for measures depending on an arbitrary number of hyper parameters.
    \item The format should unambiguously classify measures by the hyper parameter(s) that were varied for the experiment.
\end{itemize}

\section{Compatibility with Other Techniques}
\begin{itemize}
    \item The file format should allow the storage of all relevant measures during an experiment (fluorescence, Raman, etc.).
    \item All Brillouin data should be stored in a single group to allow other complimentary techniques used in an experiment to be stored in complimentary groups.
\end{itemize}

\end{document}