\documentclass[a4paper,12pt]{article}

% Packages
\usepackage[utf8]{inputenc} % Encoding
\usepackage[T1]{fontenc}  % Font encoding
\usepackage[colorlinks=true,linkcolor=blue,urlcolor=blue,citecolor=blue]{hyperref}     % Hyperlinks
\usepackage{geometry}     % Page layout
\usepackage{listings}     % Code listings
\usepackage{xcolor}       % Colors for code
\usepackage{tcolorbox} % Add this to your preamble

% Page layout
\geometry{margin=1in}

% Functions
\tcbset{
    attentionbox/.style={
        colback=red!5, % Background color
        colframe=red!75!black, % Border color
        coltitle=black, % Title color
        boxrule=0.8mm, % Thickness of the border
        left=2mm, % Left padding
        before skip=5mm, % Space before the box
        after skip=5mm, % Space after the box
        sharp corners, % Sharp corners for the box
        fonttitle=\bfseries, % Bold title font
        attach boxed title to top left={yshift=-2mm,xshift=2mm}, % Title position
        boxed title style={
            size=small,
            colback=red!50,
            colframe=red!75!black,
            sharp corners,
        },
    }
}

% Code formatting
\lstset{
    basicstyle=\ttfamily\small,
    breaklines=true,
    frame=single,
    keywordstyle=\color{blue},
    commentstyle=\color{green!70!black},
    stringstyle=\color{red},
    showstringspaces=false,
    numbers=left,
    numberstyle=\tiny\color{gray},
    stepnumber=1,
    numbersep=5pt,
}

% Metadata
\title{Developer Guide for Project \texttt{HDF5\_BLS}}
\author{Pierre Bouvet}
\date{\today}

\begin{document}

\maketitle

\tableofcontents

\section{Introduction}
This document serves as the developer guide for the modules of the\texttt{HDF5\_BLS} project. It provides details about the approach to defining new functions and integrating them into the project to ensure code reusability and maintainability and allow an adaptation to new needs.

\section{Module Structure}
The project has the following structure:
\begin{verbatim}
HDF5_BLS.
|-- load_data
|-- wrapper
|-- treat
\end{verbatim}

\subsection{Key Components}
\begin{itemize}
    \item \textbf{load\_data}: Allows the import of data from different formats.
    \item \textbf{wrapper}: Contains the Wrapper object that allows the manipulation of the data and its attributes.
    \item \textbf{treat}: Contains the treatment functions.
\end{itemize}

\section{Importing new formats}
To import a new format, you need to create a new function in the \texttt{load\_data} module. This function should take the file path as input and return the data and its extracted attributes. The function should have the following structure:
\begin{verbatim}
    def load_file_format(filepath):
        data = # Load data from the file
        attributes = # Extract attributes from the file
        return data, attributes
\end{verbatim}

\subsection{Data format}
The format of the data should be a numpy array with the last dimension being the number of spectral channels. For example: a 2D raster scan with 200 x points and 100 y points acquired with a spectrometer with 512 spectral channels would be represented as a numpy array of shape (200, 100, 512).

\subsection{Attributes format}
The attributes is a dictionary. The name of the keys of the dictionnary should match the list given in the \texttt{spreadsheets/attributes\_v01.xlsx} file. The values should be strings.

Additionnaly, it is possible to add new attributes to the data. To do so, you need to add a new key to the attributes dictionary with the following structure:
\begin{verbatim}
    attributes['CATEGORY.Attribute'] = # Value of the new attribute
\end{verbatim}
where \texttt{CATEGORY} is either \texttt{SPECTROMETER} (if its spectrometer related), \texttt{MEASURE} (if its measure related) or \texttt{FILEPROP} (if its file related), and \texttt{Attribute} is the name of the attribute. It is recommended to use the same name as the key in the spreadsheet, and update the spreadsheet with the new attribute. Additionnaly, we recommend to send a pull request to the repository to update the spreadsheet so as to normalize the naming convention in the community.

\section{Contact}
For questions or suggestions, please contact the maintainer at \href{mailto:pierre.bouvet@meduniwien.ac.at}{pierre.bouvet@meduniwien.ac.at}.

\end{document}
