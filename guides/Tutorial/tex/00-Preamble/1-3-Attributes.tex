\subsubsection*{Storing the attributes of the data in its metadata}

    HDF5 file format allows the storage of attributes in the metadata of the groups and datasets. We therefore propose to store all the attributes concerning an experiment in the metadata of its parent group:
    \begin{verbatim}
        file.h5
        +-- Brillouin (group)
        |   +-- Measure (group) -> attributes of the measure
        |   |   +-- Measure (dataset) 
    \end{verbatim}

    Being a hierarchical format, we also propose to store attributes hierarchically: all attributes of parent group apply to childre groups (if not redefined in children groups). Storing attributes in large files can therefore be done the following way:
    \begin{verbatim}
        file.h5
        +-- Brillouin (group) -> attributes shared by Measure 0 and Measure 1
        |   +-- Measure 0 (group) -> other attributes specific to Measure 0
        |   |   +-- Measure (dataset)
        |   +-- Measure 1 (group) -> other attributes specific to Measure 1
        |   |   +-- Measure (dataset)
    \end{verbatim}

    Note that the access to the whole list of attributes applying to a group or dataset will be possible with the HDF5\_BLS package (see \hyperref[subchapter:wrapper.get_attributes]{Wrapper.get\_attributes}).

\subsubsection*{Types of attributes}

    In an effort to avoid any incompatibility, we propose to store the values of the attributes as ascii-encoded text. The library will then convert the strings to the appropriate type (e.g. float, int, etc.).

\subsubsection*{Organization of the attributes}

    \paragraph*{Prefix}
    We differentiate 5 types of attributes, that we differentiate using the following prefixes:
    \begin{itemize}
        \item \textbf{SPECTROMETER} - Attributes that are specific to the spectrometer used, such as the wavelength of the laser, the type of laser, the type of detector, etc. These attributes are recognized by the capital letter word "SPECTROMETER" in the name of the attribute.
        \item \textbf{MEASURE} - Attributes that are specific to the sample, such as the date of the measure, the name of the sample, etc. These attributes are recognized by the capital letter word "MEASURE" in the name of the attribute.
        \item \textbf{FILEPROP} - Attributes that are specific to the original file format, such as the name of the file, the date of the file, the version of the file, the precision used on the storage of the data, etc. These attributes are recognized by the capital letter word "FILEPROP" in the name of the attribute.
        \item \textbf{PROCESS} - Attributes that are specific to the storage of algorithms. These attributes are recognized by the capital letter word "PROCESS" in the name of the attribute.
        \item Attributes that are used inside the HDF5 file, such as the "Brillouin\_type" attribute. These attributes are the only ones without a prefix.
    \end{itemize}

    \paragraph*{Units}
    The name of the attributes contains the unit of the attribute if it has units, in the shape of an underscore followed by the unit in parenthesis. Some parameters will also be given following a given norm, such as the ISO8601 for dates. These norms are not specified in the name of the attribute. Here are some examples of attributes:
    \begin{itemize}
        \item "SPECTROMETER.Detector\_Type" is the type of the detector used.
        \item "MEASURE.Sample" is the name of the sample.
        \item "MEASURE.Exposure\_(s)" is the exposure of the sample given in seconds
        \item "MEASURE.Date\_of\_measurement" is the date of the measurement, stored following the ISO8601 norm.
        \item "FILEPROP.Name" is the name of the file.
    \end{itemize}

\subsubsection*{Unifification and Versioning of attributes}
    To unify the name of attributes between laboratories, we propose to use a spreadsheet that contains the list of attributes, their definition, their unit and an example of value. This spreadsheet is available on the project repository and is updated as new attributes are added to the project. Each attribute has a version number that is also stored in the attributes of each data attribute (under FILEPROP.version).

    This spreadsheet will also be the preferred way to define attributes for the measures and the HDF5\_BLS package allows to read and import the attributes directly from this spreadsheet (see \hyperref[subchapter:wrapper.import_properties_data]{Wrapper.import\_properties\_data}).
    
\subsubsection*{Storing analysis and treatment processes}

    Analysis and treatment processes are stored in the "PROCESS" attribute of the treatment groups. This attribute is a JSON file converted to a string, which contains the list of treatment steps performed on the data. This JSON file has the following structure:
    \begin{verbatim}
{
    "name": "The name of the algorithm",
    "version": "v 0.1",
    "author": "Author name and affiliation",
    "description": "The description of the algorithm",
    "functions": [
    {
        "function": "The 1st function name in the class",
        "parameters": {
            "parameter_1": value,
            "parameter_2": value,
            ...
        },
        "description": "The description of the function"
    },
    {
        "function": "The 2nd function name in the class",
        "parameters": {
            "parameter_1": value,
            "parameter_2": value,
            ...
        },
        "description": "The description of the function"
    },
    ...
    ]
}
    \end{verbatim}

    When the treatment is performed using the modules of the HDF5\_BLS package, this attribute is automatically updated. Note that custom treatments can also be stored in this attribute by the user.

    This attribute can be exported to a standalone JSON file using the library. This attribute also allows the library to re-apply the treatment to the data, and modify steps of the treatment if needed.
    