\paragraph{Storing the attributes of the data in its metadata}

    HDF5 file format allows the storage of attributes in the metadata of the groups and datasets. We threfore propose to store all the attributes concerning an experiment in the metadata of its parent group:
    \begin{verbatim}
        file.h5
        +-- Brillouin (group)
        |   +-- Measure (group) -> attributes of the measure
        |   |   +-- Measure (dataset) 
    \end{verbatim}

    Being a hierarchical format, we also propose to store attributes hierarchically: all attributes of parent group apply to childre groups (if not redefined in children groups). Storing attributes in large files can therefore be done the following way:
    \begin{verbatim}
        file.h5
        +-- Brillouin (group) -> attributes shared by Measure 0 and Measure 1
        |   +-- Measure 0 (group) -> other attributes specific to Measure 0
        |   |   +-- Measure (dataset)
        |   +-- Measure 1 (group) -> other attributes specific to Measure 1
        |   |   +-- Measure (dataset)
    \end{verbatim}

    Note that the access to the whole list of attributes applying to a group or dataset will be possible with the HDF5\_BLS package (see \hyperref[subsec:wrapper.get_attributes]{Wrapper.get\_attributes}).

\paragraph{Types of attributes}

    In an effort to avoid any incompatibility, we propose to store the values of the attributes as ascii-encoded text. The library will then convert the strings to the appropriate type (e.g. float, int, etc.).

\subsubsection*{Organization of the attributes}

    \paragraph{Prefix}
    We differentiate 4 types of attributes, that we differentiate using the following prefixes:
    \begin{itemize}
        \item \textbf{SPECTROMETER} - Attributes that are specific to the spectrometer used, such as the wavelength of the laser, the type of laser, the type of detector, etc. These attributes are recognized by the capital letter word "SPECTROMETER" in the name of the attribute.
        \item \textbf{MEASURE} - Attributes that are specific to the sample, such as the date of the measurement, the name of the sample, etc. These attributes are recognized by the capital letter word "MEASURE" in the name of the attribute.
        \item \textbf{FILEPROP} - Attributes that are specific to the original file format, such as the name of the file, the date of the file, the version of the file, the precision used on the storage of the data, etc. These attributes are recognized by the capital letter word "FILEPROP" in the name of the attribute.
        \item Attributes that are used inside the HDF5 file, such as the "Brillouin\_type" attribute. These attributes are the only ones without a prefix.
    \end{itemize}

    \paragraph{Units}
    The name of the attributes contains the unit of the attribute if it has units, in the shape of an underscore followed by the unit in parenthesis. Some parameters that can be represented by a series of norms will also be defined in a given norm, such as the ISO8601 for dates. These norms are however not specified in the name of the attribute. Here are some examples of attributes:
    \begin{itemize}
        \item "SPECTROMETER.Detector\_Type" is the type of the detector used.
        \item "MEASURE.Sample" is the name of the sample.
        \item "MEASURE.Exposure\_(s)" is the exposire of the sample given in seconds
        \item "MEASURE.Date\_of\_measurement" is the date of the measurement, stored following the ISO8601 norm.
        \item "FILEPROP.Name" is the name of the file.
    \end{itemize}

\subsubsection*{Versioning and Nomenclature}
    To unify the name of attributes between laboratories, a spreadsheet is accesible, containing a list of attributes name, definition, unit and example. This spreadsheet will be updated as new attributes are added to the project and defined with a version number that will also be stored in the attribtutes of each data attributes (under FILEPROP.version). 
    
    This spreadsheet is the preferred way to define the attributes of the data and the HDF5\_BLS package is able to import the attributes from this spreadsheet to populate the attributes of the data in the file (see \hyperref[subsec:wrapper.import_properties_data]{Wrapper.import\_properties\_data}).
    
\subsubsection*{Storing analysis and treatment processes}

    Analysis and treatment processes are stored in the "Process" attribute of the treatment groups. This attribute is a text file written as a CSV file. Each treatment step is stored as a new line in the file, with the following format:
    \begin{verbatim}
"Step name or function", "Step description", "Step parameters"
    \end{verbatim}

    For example:
    \begin{verbatim}
"normalize", "Normalizing the data using the RMS of noise as the offset and highest peak 
    as intensity", "None"
"fit", "Fitting a Lorentzian to the anti-Stokes peak with initial parameters: shift and 
    linewidth, using a LDSM regression algorithm", "shift=5.143, linewidth=0.83"
"fit", "Fitting a Lorentzian to the Stokes peak with initial parameters: shift and 
    linewidth, using a LDSM regression algorithm", "shift=-5.143, linewidth=0.83"
    \end{verbatim}

    When the treatment is performed from the HDF5\_BLS package, this attribute is automatically updated. However, when a custom treatment is performed, the user is responsible for updating this attribute.

    This attribute if present and defined using functions internal to the HDF5\_BLS package, will allow the user to modify a treatment process.
    