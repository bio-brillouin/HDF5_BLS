It can be useful to store in a same file, measures coming from different instruments, taken in different conditions, or that we just want to separate from other groups of measures. In that end, we propose a tree-like structure of the HDF5 file, where each group can contain sub-groups, which can also contain sub-groups etc. In order to unify the way we access these groups, we propose to identify them by a unique identifier of the form "Data\_i", where "i" is an integer. Here is an example of the structure of a meta-file:

\begin{verbatim}
    file.h5
    +-- Data
    |   +-- Data_1
    |   |   +-- Data_1
    |   |   |   +-- ...
    |   |   +-- Data_2
    |   |   |   +-- ...
    |   |   +-- ...
    |   +-- Data_2
    |   |   +-- Data_1
    |   |   |   +-- Data_1
    |   |   |   |   +-- ...
    |   |   |   +-- Data_2
    |   |   |   |   +-- ...
    |   |   |   ...
    |   |   ...
    |   ...
\end{verbatim}