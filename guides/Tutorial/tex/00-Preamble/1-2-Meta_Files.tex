The Hierarchical Data Format (HDF5) finds its interest in storage for our community, of different measures. These result in "meta-files" where data corresponding to different experiments can be found. The organization of such a file will follow a structure similar to this one:

\begin{verbatim}
    file.h5
    +-- Brillouin (group)
    |   +-- RWPE1 organoids (group)
    |   |   +-- Morphogenesis day 1 (group)
    |   |   |   +-- Sample 1 (group)
    |   |   |   |   +-- Measure (dataset)
    |   |   |   +-- Sample 2 (group)
    |   |   |   +-- ...
    |   |   +-- Morphogenesis day 2 (group)
    |   |   +-- ...
    |   +-- H6C7 organoids
    |   ...
\end{verbatim}

Although no rules are imposed on the way to organize the file, we propose to associate a hierarchical level to an hyperparameter that has been varied for the experiment. In the given example:
\begin{itemize}
    \item The first hierarchical level is associated to the cell line that is observed
    \item The second hierarchical level is associated to the day of the experiment
    \item The third hierarchical level is associated to the sample that was measured
\end{itemize}

Note that the format does not impose any restriction on the names of the groups nor the measures. This choice allows you to create user-friendly files that can be opened with any software that can read HDF5 files (e.g. HDFView, HDFCompass, Fiji, H5Web, etc.).