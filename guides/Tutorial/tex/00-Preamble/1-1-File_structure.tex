This project aims at defining a standard for storing Brillouin Light Scattering measures and associated treatment in a HDF5 file.

HDF5 stands for "Hierarchical Data Format" and is a file format that allows the storage of data in a hierarchical structure. This structure allows to store data in a way that is both human and machine readable. The structure of the file is based on the following base structure, which corresponds to the structure of a file containing a single measure (\textit{Measure}) where no parameters have been stored:

\begin{verbatim}
    file.h5
    +-- Brillouin (group)
    |   +-- Measure (group)
    |   |   +-- Measure (dataset)
\end{verbatim}

The dimensionality of the dataset is free, there are therefore by design virtually no restrictions on the data that can be stored in this format.

The organization of the file is based on the following principles:
\begin{itemize}
    \item The file is organized in groups and datasets, which allows to store data in a hierarchical structure.
    \item Only one dataset corresponding to a measure can be stored per group. 
    \item The groups are used to organize the file and store metadata and parameters related to the measure, and the datasets are used to store the actual data.
\end{itemize}