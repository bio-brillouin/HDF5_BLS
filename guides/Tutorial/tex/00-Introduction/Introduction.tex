Welcome to the HDF5\_BLS tutorial. This document is designed to help you get started with the HDF5\_BLS package.

The HDF5\_BLS package is a Python library for handling Brillouin light scattering (BLS) data and converting it into a standardized HDF5 format. The library provides functions to open raw data files, store their properties, convert them into a Power Spectral Density (PSD) and analyze the PSD with a standardized treatment protocol. The library is currently compatible with the following file formats:
\begin{itemize}
    \item \textbf{*.dat} files: spectra returned by the GHOST software or obtained using Time Domain measurements
    \item \textbf{*.tif} files: an image format that can be used to export 2D detector images.
    \item \textbf{*.npy} files: an arbitrary numpy array
    \item \textbf{*.sif} files: image files obtained with Andor cameras
\end{itemize}

The package comes with a graphical user interface (GUI) that allows users to easily open, edit, and save data. This interface is the preferred way to use the package and the subject of this tutorial. The GUI is currently compatible with the following spectrometers:
\begin{itemize}
    \item Tandem Fabry-Perot (TFP) spectrometers
    \item 1-VIPA spectrometers
    \item Angle-resolved VIPA (ar-VIPA) spectrometers 
    \item Time-domain spectrometers
\end{itemize}