This method allows the user to update one attribute of the wrapper. This method is based on the \hyperref[subchapter:wrapper.set_attributes_data]{Wrapper.set\_attributes\_data} method.

\begin{lstlisting}[language=Python]
def update_property(self, name, value, path = None):
\end{lstlisting}

\paragraph{Attributes:}

\begin{itemize}
    \item \textbf{name}: The name of the attribute to update.
    \item \textbf{value}: The new value of the attribute.
    \item \textbf{path} \textit{(optional, default None)}: The path to the group or dataset whose metadata we want to update in the form: "Brillouin/Measure". If None, the metadata of the root group are updated.
\end{itemize}

\paragraph{Raises:}
\begin{itemize}
    \item All the errors of the \hyperref[subchapter:wrapper.set_attributes_data]{Wrapper.set\_attributes\_data} method.
\end{itemize}


\paragraph{Flowchart:}

The function's logic is represented in the following flowchart:
\begin{figure}[H]
    \centering
    \label{fig:wrapper.flowchart_update_property}
    \small
    \begin{tikzpicture}[node distance=1.5cm]
        \node (start) [startstop, xshift=0cm, align=center] {Wrapper.update\_property(name, \\value, path)};
        \node (setProperty) [process, below of=start, align=center] {Wrapper.set\_properties\_data(properties={name: value},\\ path=path, overwrite=True)};
        \node (end) [startstop, below of=setProperty, xshift=0cm, align=center] {End};

        \draw [arrow] (start) -- (setProperty);
        \draw [arrow] (setProperty) -- (end);
    \end{tikzpicture}
    \caption{Flowchart of the import\_properties\_data method}
\end{figure}