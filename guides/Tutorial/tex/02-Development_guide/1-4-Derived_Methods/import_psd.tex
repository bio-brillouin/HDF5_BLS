This method allows the user to add a Power Spectral Density array to the HDF5 file from a file. It adds the data by calling the \hyperref[subchapter:wrapper.add_psd]{Wrapper.add\_PSD} method.


\begin{lstlisting}[language=Python]
def import_PSD(self, filepath, parent_group, creator = None, parameters = None, name = None, reshape = None, overwrite = False):
\end{lstlisting}

\paragraph{Attributes:}

\begin{itemize}
    \item \textbf{filepath}: The path to the file containing the PSD to import.
    \item \textbf{parent\_group}: The path to the group or dataset where the frequency will be added in the form "Brillouin/Measure/...".
    \item \textbf{creator} \textit{(optional, default None)}: The structure of the file that has to be loaded. If None, a LoadError can be raised.
    \item \textbf{parameters} \textit{(optional, default None)}: The parameters that are to be used to import the data correctly.  If None, a LoadError can be raised.
    \item \textbf{name} \textit{(optional, default None)}: The name that will be given to the dataset. If None, the name is set to "PSD".
    \item \textbf{reshape} \textit{(optional, default None)}: The new shape of the array. If None, the shape is not changed.
    \item \textbf{overwrite} \textit{(optional, default False)}: If True, the attributes of the selected group or dataset are overwritten if they exist in the file.
\end{itemize}


\paragraph{Raises:}
\begin{itemize}
    \item \textbf{WrapperError\_FileNotFound}: If the file could not be found.
    \item All the errors of the \hyperref[subchapter:wrapper.add_psd]{Wrapper.add\_PSD} method.
    \item All the LoadError errors of the load\_data module.
\end{itemize}