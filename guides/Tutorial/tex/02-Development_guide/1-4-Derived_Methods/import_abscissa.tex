This method allows the user to add an abscissa to the HDF5 file from a measure file. It adds the data by calling the \hyperref[subchapter:wrapper.add_abscissa]{Wrapper.add\_abscissa} method.

\begin{lstlisting}[language=Python]
def add_abscissa(self, filepath, parent_group, creator = None, parameters = None, name=None, unit = "AU" , dim_start = 0, dim_end = None, reshape = None, overwrite = False):
\end{lstlisting}

\paragraph{Attributes:}

\begin{itemize}
    \item \textbf{filepath}: The path to the file containing the abscissa to import.
    \item \textbf{parent\_group}: The path to the group or dataset whose the raw data will be added in the form "Brillouin/Measure/...".
    \item \textbf{creator} \textit{(optional, default None)}: The structure of the file that has to be loaded. If None, a LoadError can be raised.
    \item \textbf{parameters} \textit{(optional, default None)}: The parameters that are to be used to import the data correctly.  If None, a LoadError can be raised.
    \item \textbf{name} \textit{(optional, default None)}: The name that will be given to the dataset. If None, the name is set to "Raw data".
    \item \textbf{unit} \textit{(optional, default "1")}: The unit of the abscissa (e.g. "microns").
    \item \textbf{dim\_start} \textit{(optional, default 0)}: The first dimension of the abscissa.
    \item \textbf{dim\_end} \textit{(optional, default None)}: The last dimension of the abscissa. If None, the abscissa is considered to be a vector.
    \item \textbf{reshape} \textit{(optional, default None)}: The new shape of the array. If None, the shape is not changed.
    \item \textbf{overwrite} \textit{(optional, default False)}: If True, the attributes of the selected group or dataset are overwritten if they exist in the file.
\end{itemize}

\paragraph{Raises:}
\begin{itemize}
    \item \textbf{WrapperError\_FileNotFound}: If the file could not be found.  
    \item All the errors of the \hyperref[subchapter:wrapper.add_abscissa]{Wrapper.add\_abscissa} method.
    \item All the LoadError errors of the load\_data module.
\end{itemize}
