This method allows the user to add the result of a treatment to the HDF5 file. It adds the data by calling the \hyperref[subchapter:wrapper.add_dictionary]{Wrapper.add\_dictionary} method.

\begin{lstlisting}[language=Python]
def add_treated_data(self, shift, linewidth, shift_err, linewidth_err, parent_group, name_group = None, overwrite = False):
\end{lstlisting}

\paragraph{Attributes:}

\begin{itemize}
    \item \textbf{shift}: The shift array to add to the wrapper. 
    \item \textbf{linewidth}: The linewidth array to add to the wrapper. 
    \item \textbf{shift\_err}: The error of the shift array to add to the wrapper. 
    \item \textbf{linewidth\_err}: The error of the linewidth array to add to the wrapper. 
    \item \textbf{parent\_group}: The path to the group or dataset whose the raw data will be added in the form "Brillouin/Measure/...".
    \item \textbf{name\_group} \textit{(optional, default None)}: The name that will be given to the group containing all the treated dataset. If None, the name is set to "Treat\_i" where "i" is an integer that ensures that the name is unique.
    \item \textbf{overwrite} \textit{(optional, default False)}: If True, the attributes of the selected group or dataset are overwritten if they exist in the file.
\end{itemize}

\paragraph{Raises:}
\begin{itemize}
    \item \textbf{WrapperError\_StructureError}: If the parent group does not exist in the HDF5 file.
    \item All the errors of the \hyperref[subchapter:wrapper.add_dictionary]{Wrapper.add\_dictionary} method.
\end{itemize}
