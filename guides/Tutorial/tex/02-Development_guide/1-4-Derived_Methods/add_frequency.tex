This method allows the user to add a frequency array to the HDF5 file. It adds the data by calling the \hyperref[subsec:wrapper.add_dictionnary]{Wrapper.add\_dictionnary} method.
By default, the frequency array is stored in "GHz". Note however that this will just affect the presentation of the results and not the process itself.

\begin{lstlisting}[language=Python]
def add_frequency(self, data, parent_group, name = None, overwrite = False):
\end{lstlisting}

\paragraph{Attributes:}

\begin{itemize}
    \item \textbf{data}: The frequency array to add to the wrapper. 
    \item \textbf{parent\_group}: The path to the group or dataset where the frequency will be added in the form "Brillouin/Measure/...".
    \item \textbf{name} \textit{(optional, default None)}: The name that will be given to the dataset. If None, the name is set to "Frequency".
    \item \textbf{overwrite} \textit{(optional, default False)}: If True, the attributes of the selected group or dataset are overwritten if they exist in the file.
\end{itemize}

\paragraph{Raises:}
\begin{itemize}
    \item \textbf{WrapperError\_StructureError}: If the parent group does not exist in the HDF5 file.
    \item All the errors of the \hyperref[subsec:wrapper.add_dictionnary]{Wrapper.add\_dictionnary} method.
\end{itemize}
