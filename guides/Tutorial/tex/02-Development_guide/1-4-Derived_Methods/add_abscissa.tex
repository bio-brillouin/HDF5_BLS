This method allows the user to add an abscissa to the HDF5 file. It adds the data by calling the \hyperref[subsec:wrapper.add_dictionnary]{Wrapper.add\_dictionnary} method.

\begin{lstlisting}[language=Python]
def add_abscissa(self, data, parent_group=None, name=None, unit = "1" , dim_start = 0, dim_end = None, overwrite = False):
\end{lstlisting}

\paragraph{Attributes:}

\begin{itemize}
    \item \textbf{data}: The abscissa array to add to the wrapper. 
    \item \textbf{parent\_group}: The path to the group or dataset whose the raw data will be added in the form "Brillouin/Measure/...".
    \item \textbf{name} \textit{(optional, default None)}: The name that will be given to the dataset. If None, the name is set to "Raw data".
    \item \textbf{unit} \textit{(optional, default "1")}: The unit of the abscissa (e.g. "microns").
    \item \textbf{dim\_start} \textit{(optional, default 0)}: The first dimension of the abscissa.
    \item \textbf{dim\_end} \textit{(optional, default None)}: The last dimension of the abscissa. If None, the abscissa is considered to be a vector.
    \item \textbf{overwrite} \textit{(optional, default False)}: If True, the attributes of the selected group or dataset are overwritten if they exist in the file.
\end{itemize}

\paragraph{Raises:}
\begin{itemize}
    \item All the errors of the \hyperref[subsec:wrapper.add_dictionnary]{Wrapper.add\_dictionnary} method.
\end{itemize}
