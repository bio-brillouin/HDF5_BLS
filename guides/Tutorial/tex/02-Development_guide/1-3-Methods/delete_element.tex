This method allows the user to delete a dataset or group in the HDF5 file.

\begin{lstlisting}[language=Python]
def delete_element(self, path = None):
\end{lstlisting}

\paragraph{Attributes:}

\begin{itemize}
    \item \textbf{path} \textit{(optional, default None)}: The path to the group or dataset we want to delete in the form "Brillouin/Measure/PSD". If None, the whole file is deleted.
\end{itemize}

\paragraph{Raises:}
\begin{itemize}
    \item \textbf{WrapperError\_StructureError}: If the path does not lead to a valid element in the file.
    \item \textbf{WrapperError}: If there are errors in the deleting process.
\end{itemize}


\paragraph{Flowchart:}

The function's logic is represented in the following flowchart:
\begin{figure}[H]
    \centering
    \label{fig:wrapper.flowchart_delete_element}
    \small
    \begin{tikzpicture}[node distance=1cm]
        \node (start) [startstop, xshift=0cm, align=center] {Wrapper.delete\_element(self, path):};
        \node (setPath) [process, below of=start, yshift=-1cm, xshift=-4cm, align=center] {path = "Brillouin"};
        \node (raiseError1) [startstop, right of=start, xshift=4cm, yshift=-1cm] {WrapperError\_StructureError};
        \node (delete) [process, below of=start, yshift=-3cm, xshift=0cm, align=center] {Delete the element};
        \node (raiseError2) [startstop, below of=raiseError1, xshift=0cm, yshift=-0.5cm] {WrapperError};
        \node (end) [startstop, below of=raiseError2, xshift=0cm, yshift=-0.5cm, align=center] {End};

        \draw [arrow] (start) |- node[anchor=south, align=center, yshift=0cm, xshift=-1cm] {path is\\ None} (setPath);
        \draw [arrow] (start) -| node[anchor=south, align=center, yshift=0cm, xshift=0cm] {path doesn't lead\\to an element} (raiseError1);
        \draw [arrow] (setPath) |- node[anchor=north, align=center, yshift=0cm, xshift=-2.5cm] {For each element\\ in "Brillouin" group:\\ path = "Brillouin/"+element} (start);
        \draw [arrow] (start) -- node[anchor=east, align=center, yshift=-1cm, xshift=0cm] {path leads to\\ an element} (delete);
        \draw [arrow] ++(delete.east) -| +(1cm,1cm) |- node[anchor=east, align=center, yshift=0.5cm, xshift=1cm] {Error occured while \\ deleting the element} (raiseError2.west);
        \draw [arrow] (delete) -- node[anchor=north, align=center, yshift=0cm, xshift=0cm] {else} (end);
    \end{tikzpicture}
    \caption{Flowchart of the delete\_element method}
\end{figure}