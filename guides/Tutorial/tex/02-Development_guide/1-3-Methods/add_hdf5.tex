This method allows the user to add an HDF5 file to the file under a specific group. The group is created if it does not exist. The attributes of the HDF5 file are only added to the created group if they are different from the parent's attribute.

\begin{lstlisting}
def add_hdf5(self, filepath, parent_group = None, overwrite = False):
\end{lstlisting}

\paragraph{Attributes:}

\begin{itemize}
    \item \textbf{filepath}: The path to the HDF5 file to add.
    \item \textbf{parent\_group} \textit{(optional, default None)}: The parent group where to store the data of the HDF5 file, by default the parent group is the top group "Brillouin". The format of this group should be "Brillouin/Measure/...". 
    \item \textbf{overwrite} \textit{(optional, default False)}: If True, the group of same name than the HDF5 file that is added is overwritten. If False, only new attributes are added to the existing ones.
\end{itemize}

\paragraph{Returns:} Nothing

\paragraph{Raises:}

\begin{itemize}
    \item \textbf{WrapperError\_FileNotFound}: Raises an error if the filepath of the HDF5 file does not exist.
    \item \textbf{WrapperError\_StructureError}: Raises an error if the parent group does not exist in the HDF5 file.
    \item \textbf{WrapperError\_Overwrite}: Raises an error if the file name already exists in the selected group.
    \item \textbf{WrapperError}: Raises an error if the addition of the HDF5 file failed.
\end{itemize}

\paragraph{Flowchart:}

The function's logic is represented in the following flowchart:
\begin{figure}[H]
    \centering
    \label{fig:wrapper.flowchart_add_hdf5}
    \small
    \begin{tikzpicture}[node distance=2cm]
        \node (start) [startstop, align=center] {Wrapper.add\_hdf5(\\filepath, parent\_group)};
        \node (raiseError0) [startstop, right of=start, xshift=6cm] {WrapperError\_FileNotFound};
        \node (setParent) [process, below of=start, yshift=-0.5cm, xshift=-3cm] {parent\_group = "Brillouin"};
        \node (checkExists) [process, right of=setParent, xshift=3cm, align=center] {Check if parent\_group\\ exists in HDF5 file};
        \node (raiseError1) [startstop, right of=checkExists, xshift=4cm] {WrapperError\_StructureError};
        \node (checkGroup) [process, below of=checkExists, yshift=0.5cm, align=center] {Check if parent\_group\\ is a group};
        \node (getGroup) [process, right of=checkGroup, yshift=-1.5cm, xshift = 2cm, align=center] {Get the group\\above parent\_group};
        \node (checkName) [process, below of=checkGroup, yshift=-1cm, align=center] {Check if file name\\  already exists in\\ selected group};
        \node (raiseError2) [startstop, right of=checkName, xshift=3cm] {WrapperError\_Overwrite};
        \node (addData) [process, below of=setParent, yshift=-2.5cm, align=center] {Add HDF5 file to file\\as a new group};
        \node (raiseError3) [startstop, below of=addData, yshift=-2cm] {WrapperError};
        \node (end) [startstop, below of=addData, xshift=5cm, yshift=0cm,] {End};

        \draw [arrow] (start) -- node[anchor=south] {filepath not valid} (raiseError0);
        \draw [arrow] (start) -| node[anchor=east, align=center, yshift=-1cm] {filepath valid \\ parent\_group \\is None} (setParent);
        \draw [arrow] ++(start.south) -- +(0mm, -1cm) -| node[anchor=south, align=center] {filepath valid \\ parent\_group is not None} (checkExists.north);
        \draw [arrow] (checkExists) -- node[anchor=north,align=center] {doesn't \\exist} (raiseError1);
        \draw [arrow] (checkExists) -- (checkGroup);
        \draw [arrow] (checkGroup) -- node[anchor=east] {is group} (checkName);
        \draw [arrow] (checkGroup) -| node[anchor=west] {is not group} (getGroup);
        \draw [arrow] (getGroup) -| (checkName);
        \draw [arrow] (checkName) -- node[anchor=south] {exists} (raiseError2);
        \draw [arrow] ++(checkName.west) -| +(-1cm, 0cm) |- node[anchor=south, align=center] {doesn't\\ exist} (addData.east);
        \draw [arrow] (setParent) -- (addData);
        \draw [arrow] (addData) |- node[anchor=south, align=center, xshift = 2cm] {Success} (end);
        \draw [arrow] (addData.south) -- node[anchor=east, yshift=-1cm] {Fail} (raiseError3.north);
    \end{tikzpicture}
    \caption{Flowchart of the add\_hdf5 method}
\end{figure}