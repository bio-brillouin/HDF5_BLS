\begin{tcolorbox}
    The "Wrapper" class is the main class of the package. Its role is to interface the file and the software while ensuring a user-friendly access to the data and a conservation of the structure of the file. This class is meant to do these four actions:
    \begin{enumerate}
        \item Creates a structure that is universal to the BioBrillouin community in a seamless manner.
        \item Allow the user to acces the data and attributes of the file.
        \item Allow the user to add data to the file with attributes that are specific to the BioBrillouin community.
        \item Allow the user to add or modify attributes of the file.
    \end{enumerate}
\end{tcolorbox}

\paragraph{Memory management:}
HDF5 files are notorious for behing heavy. Storing a whole HDF5 file in flash memory is therefore generally a bad idea. As such, the "Wrapper" class works by accessing a file on the disk. By default this file is a temporary file stored in the project's directory. The class is however designed to work on existing HDF5 files stored in permanent locations.

\paragraph{Private attributes:} 
At initialization:
\begin{itemize}
    \item \textbf{self.filepath}: The path to the HDF5 file. By default, the path to the temporary file is used.
    \item \textbf{self.save}: A boolean that indicates whether the file has been saved or is still in the temporary file.
\end{itemize}

\paragraph{Dunders:} 
The "Wrapper" class has the following dunder methods that are described in the following sections:
\begin{itemize}
    \item \hyperref[subsec:wrapper.__init__]{Wrapper.\_\_init\_\_ -> Wrapper()}: The method that initializes the object
    \item \hyperref[subsec:wrapper.__getitem__]{Wrapper.\_\_getitem\_\_ -> Wrapper[key]}: The preferred method to access an element located at a given path, it allows to access an element by placing its path in the brackets of the "Wrapper" object.
    \item \hyperref[subsec:wrapper.__add__]{Wrapper.\_\_add\_\_ -> Wrapper + Wrapper}: A magic command to merge two wrappers together. It merges the contents of the "Brillouin" group.
\end{itemize}

\paragraph{Main Methods:} 
The "Wrapper" class has a number of methods that will be described in the following sections. The construction of the object has been done with a bottleneck strategy, where the critical interactions with the file are done with these methods:
\begin{itemize}
    \item \hyperref[subsec:wrapper.add_hdf5]{add\_hdf5}: A method to populate the file from a HDF5 file.
    \item \hyperref[subsec:wrapper.add_dictionnary]{add\_dictionnary}: A method to populate the file from a dictionnary.
    \item \hyperref[subsec:wrapper.create_group]{create\_group}: A method to create a new group inside the HDF5 file.
    \item \hyperref[subsec:wrapper.delete_element]{delete\_element}: A method to delete a dataset or group in the HDF5 file.
    \item \hyperref[subsec:wrapper.get_attributes]{get\_attributes}: A method to extract the attributes of a group or dataset.
    \item \hyperref[subsec:wrapper.get_structure]{get\_structure}: A method to extract the structure of the file.
    \item \hyperref[subsec:wrapper.save_as_hdf5]{save\_as\_hdf5}: A method to save the file to a desired location.
    \item \hyperref[subsec:wrapper.set_attributes_data]{set\_attributes\_data}: A method to set the attributes of a group or dataset.
\end{itemize}

\paragraph{Derived Methods:} 
To simplify however the creation and use of the file, these other methods also exist:
\begin{itemize}
    \item \hyperref[subsec:wrapper.add_abscissa]{add\_abscissa}: A method to add an abscissa arrayto the file.
    \item \hyperref[subsec:wrapper.add_frequency]{add\_frequency}: A method to add a frequency array to the file.
    \item \hyperref[subsec:wrapper.add_psd]{add\_psd}: A method to add a PSD to the file.
    \item \hyperref[subsec:wrapper.add_raw_data]{add\_raw\_data}: A method to add raw data to the file.
    \item \hyperref[subsec:wrapper.add_treated_data]{add\_treated\_data}: A method to add a shift, linewidth and their respective standard deviations to the file.
    \item \hyperref[subsec:wrapper.import_properties_data]{import\_properties\_data}: A method to import the attributes of a dataset or group from a spreadsheet.
    \item \hyperref[subsec:wrapper.update_property]{update\_property}: A method to update the attributes of a dataset or group.
\end{itemize}

\paragraph{Console specific Methods:} 
Additionnaly, the "Wrapper" class has a few methods specifically to interact with the file from a terminal:
\begin{itemize}
    \item \hyperref[subsec:wrapper.print_structure]{print\_structure}: A method to print a tree view of the file in the terminal.
    \item \hyperref[subsec:wrapper.print_metadata]{print\_metadata}: A method to print all the attributes of a dataset or group at a given path in the terminal.
\end{itemize}

 