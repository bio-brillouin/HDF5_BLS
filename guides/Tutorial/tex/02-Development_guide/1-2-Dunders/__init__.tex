This magic command allows the user to initialize the wrapper. This dunder method is called with the following syntax:
\begin{lstlisting}
    wrp = Wrapper(filepath)
\end{lstlisting}

\paragraph{Attributes:}

\begin{itemize}
    \item \textbf{filepath}: \textit{optional, default None} The path to the HDF5 file. By default, a temporary file is created in the project's directory.
\end{itemize}

This method is called automatically when the "Wrapper" object is created. If it has to create a HDF5 file, it automatically creates the following structure:
\begin{verbatim}
    file.h5
    +-- Brillouin (group)
\end{verbatim}

By default, the attributes of the "Brillouin" group are the following:
\begin{verbatim}
    file.h5
    +-- Brillouin (group)
    |   +-- Brillouin_type -> "Root"
    |   +-- HDF5_BLS_version -> "1.0" # The version of the package
\end{verbatim}

If the file already exists, the wrapper object just stores the path to the file and does not overwrite it.