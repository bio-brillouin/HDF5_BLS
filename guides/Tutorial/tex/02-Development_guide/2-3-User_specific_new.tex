\begin{tcolorbox}
    You are in the situation where you are using a new format that is not supported by the \texttt{HDF5\_BLS} package.
\end{tcolorbox}

Here are the steps to follow:
\begin{enumerate}
    \item Navigate to the \texttt{load\_formats} folder of the \texttt{HDF5\_BLS} package. 
    \item Create a new python file with the name of the format you are using (for example "load\_unicorn.py" if you are using ".unicorn" files).
    \item Add the function that will load your data to the file. The function should have the following signature:
\begin{lstlisting}
def load_unicorn_Wien(filepath, parameters = None):
\end{lstlisting}
    In the case where you don't need to load the data with parameters, the function should have the following signature:
\begin{lstlisting}
def load_dat_Wien(filepath):
\end{lstlisting}
    \item Write the code that will load your data. Your function should retunr a dictionary with at least two keys: "Data" and "Attributes". The "Data" key should contain the data you are loading and the "Attributes" key should contain the attributes of the file. You can also add abscissa to your data if you want to, in that case, add the key "Abscissa\_\textsl{name}" where \textsl{name} is the name you want to give to the abscissa (for example "Abscissa\_Time").
    \item Go to the \texttt{load\_data.py} file in the \texttt{HDF5\_BLS} package and create the function dedicated to the format you are using (for example "load\_unicorn\_file" if you are using ".unicorn" files)
    \item Make sure that you are importing the function you just created:
\begin{lstlisting}
from HDF5_BLS.load_formats.load_unicorn import load_unicorn_Wien
\end{lstlisting}
    \item Add a test to the function in the "tests/load\_data\_test.py" file with a test file placed in the "tests/test\_data" folder. This test is important as they are run automatically when the package is pushed to GitHub (ie: it makes my life easier ^^). 
    \item You can now use your data format with the \texttt{HDF5\_BLS} package, and in particular, the GUI. You are invited to push your code to GitHub and create a pull request to the main repository :)
\end{enumerate}
