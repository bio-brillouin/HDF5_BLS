The analysis module is built following a Object-Oriented approach. It is composed of classes called \textit{Analysis\_type} where "type" refer to the type of technique being used to acquire the data. These classes are the main classes of the module. They rely on 5 core attributes:
\begin{itemize}
    \item \textit{\_algorithm}: A JSON file describing a particular sequence of functions to be applied for the analysis
    \item \textit{x}: The abscissa of the data on which we perform the analysis
    \item \textit{y}: The data points that are to be analyzed
    \item \textit{points}: A dictionnary of points on which to base the analysis
    \item \textit{windows}: A dictionnary of windows on which to base the analysis
\end{itemize}

This chapter will present successively:
\begin{itemize}
    \item The different attributes of the class, their structure, the conditions that they must fulfill, and their intended uses (see \hyperref[sec:analysis.attributes]{section \ref{sec:analysis.attributes}}).
    \item The modular approach of the analysis module and the way functions are to be implemented. 
    \item The creation of an algorithm for reusable analysis.
    \item The versioning of the analysis module and of its functions.
    \item The testing strategy.
\end{itemize}


