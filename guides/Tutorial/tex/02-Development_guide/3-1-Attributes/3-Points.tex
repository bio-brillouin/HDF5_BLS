The \textit{points} attributes is a dictionary of points referenced in the \textit{self.x} arrays. The keys of the dictionary have a nomenclature constrained by the type of points. This nomenclature is of the following form:
\begin{center}
    prefix\_number
\end{center}
where \textit{prefix} is a string describing the type of point and \textit{number} is an integer describing the position of the point in the list of points of this type. The \textit{prefix} can be one of the following:
\begin{itemize}
    \item \textit{anti-stokes}: A point corresponding to an anti-stokes peak
    \item \textit{stokes}: A point corresponding to a stokes peak
    \item \textit{elastic}: A point corresponding to an elastic peak
    \item \textit{peak}: A point corresponding to an other peak
\end{itemize}

Example:
\begin{lstlisting}
self.points ={
    "anti-stokes\_1" = 5.5542
    "stokes\_1" = -5.4363
}
\end{lstlisting}

Note that all the position are stored as float. Note also that the keys of the \textit{points} attribute are excatly the same as the keys of the \textit{windows} attribute.
