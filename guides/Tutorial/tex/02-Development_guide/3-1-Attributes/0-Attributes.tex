All the classes of the analysis module are built following a Object-Oriented approach and rely on 5 core attributes:
\begin{itemize}
    \item \textit{\_algorithm}: A JSON file describing a particular sequence of functions to be applied for the analysis
    \item \textit{x}: The abscissa of the data on which we perform the analysis
    \item \textit{y}: The data points that are to be analyzed
    \item \textit{points}: A dictionnary of points on which to base the analysis
    \item \textit{windows}: A dictionnary of windows on which to base the analysis
\end{itemize}

Additionaly, a sixth attribute, \textit{\_history} is used to store the evolution of each attribute during the analysis. This is however a "silent" attribute that the developper should not consider as it is automatically updated as will be explained in the following sections.


\subsection{The \_algorithm attribute} \label{subsec:analysis.attributes.algorithm}
    The \textit{\_algorithm} attribute is a JSON file describing a particular sequence of functions to be applied for the analysis. It is built with two ideas in mind:
\begin{itemize}
    \item Store the sequence of functions that are applied for the analysis, with all their parameters, so that it is possible to reproduce the analysis or improve it by changing the parameters of the functions.
    \item Store a human-readable description of the sequence of functions that are applied for the analysis, so that it is possible to understand what is done in the analysis even for someone who is not familiar with the code.
\end{itemize}

The \textit{\_algorithm} attribute is a JSON file that is built with the following structure:
\begin{itemize}
    \item \textit{name}: The name of the algorithm used for the analysis
    \item \textit{version}: The version of the algorithm
    \item \textit{author}: The author of the algorithm (laboratory or person)
    \item \textit{description}: A human-readable short presentation of the algorithm
    \item \textit{functions}: A list of functions to be applied for the analysis following this structure:
    \begin{itemize}
        \item \textit{function}: The name of the function to be applied. This name does not include the name of the class as this name depends on the name of "SPECTROMETER.Type" attribute of the data to be treated.
        \item \textit{version}: The version of the function
        \item \textit{parameters}: A dictionary of parameters to be passed to the function
        \item \textit{description}: A human-readable presentation of the function
    \end{itemize}
\end{itemize}

As an example, the following JSON file describes a simple analysis that consists in taking the mean of the data points:

\begin{lstlisting}[language=json]
{
    "name": "Mean",
    "version": "1.0",
    "author": "Pierre Bouvet - Medical University of Vienna",
    "description": "Mean of the data points",
    "functions": [
        {
            "function": "mean",
            "version": "1.0",
            "parameters": {
                points: None
                }
            "description": "Mean of the data points"
        }
    ]
}
\end{lstlisting}

This description of the algorithm is "blank" meaning that the parameters are not defined. This is useful when the algorithm is stored as a sequence of functions that have to be adapted. However, if the analysis concerns a specific data set, then we can pass the attribute names as parameters, as well as any other parameters that are used for the analysis. For example, if we want to perform a polynomial fit with a polynomial of order 2 on the data points windowed around the first anti-stokes peak as defined in the "window" attribute, we can write the following JSON file:

\begin{lstlisting}[language=json]
{
    "name": "Polynomial fit",
    "version": "1.0",
    "author": "Pierre Bouvet - Medical University of Vienna",
    "description": "Polynomial fit of the data points",
    "functions": [
        {
            "function": "polyfit",
            "version": "1.0",
            "parameters": {
                "x": "self.x",
                "y": "self.y",
                "window": "self.windows['anti_stokes_1']",
                "degree": 2
                }
            "description": "Polynomial fit of the data points performed on x=self.x and y=self.y, around the window defined by self.windows['anti_stokes_1'], with a polynomial of order 2"
        }
    ]
}
\end{lstlisting}

Storing algorithms composed of a sequence of function is done by adding a function to the list of functions, following the same structure as the one described above:

\begin{lstlisting}[language=json]
    {
        "name": "Polynomial fit",
        "version": "1.0",
        "author": "Pierre Bouvet - Medical University of Vienna",
        "description": "Polynomial fit of the data points",
        "functions": [
            {
                "function": "function_1",
                "version": "1.0",
                "parameters": {
                    "x": "self.x",
                    "degree": 2
                    }
                "description": "The first function of the analysis performed on x=self.x with a degree of 2"
            },
            {
                "function": "function_2",
                "version": "1.0",
                "parameters": {
                    "y": "self.y",
                    "resampling": True
                    }
                "description": "The second function of the analysis performed on y=self.y with resampling=True"
            }
        ]
    }
\end{lstlisting}

Note that these examples are meant to present the principle at the core of the treatment but do not represent the actual implementation of the functions. The actual implementation of the functions is detailed in the following sections. Full examples will also be given at the end of this chapter.

\subsection{The x and y attributes} \label{subsec:analysis.attributes.x_and_y}
    The \textit{x} and \textit{y} attributes are respectively the abscissa and the ordinate of the data on which the analysis is performed. They are both numpy arrays of the same length. The \textit{x} attribute is the abscissa of the data points, while the \textit{y} attribute is the data points themselves. These arrays are 1D as they are plotted on a 1D graph. 

These arrays are subject to evolve during the analysis, for example if a normaliation occurs. To store their evolution and maximize the speed of the algorithm, whenever these arrays are changed, the \textit{history} attribute is updated. 

\subsection{Points} \label{subsec:analysis.attributes.points}
    The \textit{points} attributes is a dictionnary of points referenced in the \textit{self.x} arrays. The keys of the dictionnary have a nomenclature constrained by the type of points. This nomenclature is of the following form:
\begin{center}
    prefix\_number
\end{center}
where \textit{prefix} is a string describing the type of point and \textit{number} is an integer describing the position of the point in the list of points of this type. The \textit{prefix} can be one of the following:
\begin{itemize}
    \item \textit{anti-stokes}: A point corresponding to an anti-stokes peak
    \item \textit{stokes}: A point corresponding to a stokes peak
    \item \textit{elastic}: A point corresponding to an elastic peak
    \item \textit{peak}: A point corresponding to an other peak
\end{itemize}

Example:
\begin{lstlisting}
self.points ={
    "anti-stokes\_1" = 5.5542
    "stokes\_1" = -5.4363
}
\end{lstlisting}

Note that all the position are stored as float. Note also that the keys of the \textit{points} attribute are excatly the same as the keys of the \textit{windows} attribute.


\subsection{Windows} \label{subsec:analysis.attributes.windows}
    The \textit{windows} attributes is a dictionary of tuple stored as values in the \textit{self.x} array. The keys of the dictionary have a nomenclature constrained by the type of points. This nomenclature is of the following form:
\begin{center}
    prefix\_number
\end{center}
where \textit{prefix} is a string describing the type of point and \textit{number} is an integer describing the position of the point in the list of points of this type. The \textit{prefix} can be one of the following:
\begin{itemize}
    \item \textit{anti-stokes}: A point corresponding to an anti-stokes peak
    \item \textit{stokes}: A point corresponding to a stokes peak
    \item \textit{elastic}: A point corresponding to an elastic peak
    \item \textit{peak}: A point corresponding to an other peak
\end{itemize}

Example:
\begin{lstlisting}
self.windows ={
    "anti-stokes\_1" = (3.0, 8.0)
    "stokes\_1" = (-8.0, -3.0)
}
\end{lstlisting}

Note that all the position are stored as float. Note also that the keys of the \textit{points} attribute are excatly the same as the keys of the \textit{points} attribute.


\subsection{The \_history attribute} \label{subsec:analysis.attributes.history}
    The \textit{\_history} attribute resembles the \textit{algorithm} attribute in the sense that it is also a list of dictionnaries where each element corresponds to a step of the analysis. However, the main difference is that the \textit{\_history} attribute is not a JSON file but a "silent" attribute that is automatically updated whenever a function is called. This means that the \textit{\_history} attribute is not meant to be used by the user but is rather used internally by the module to store the evolution of the attributes during the analysis and speed up the analysis whenever the user decides to change a parameter during the analysis.

The elements of the \textit{\_history} attribute are dictionnaries that contain 5 keys, 4 of which are optional to minimize the size of the memory used:
\begin{itemize}
    \item \textit{function}: The algorithm used for the analysis
    \item \textit{x} (optional): The abscissa of the data on which we perform the analysis (only if the attribute has been changed at this step)
    \item \textit{y} (optional): The data points that are to be analyzed (only if the attribute has been changed at this step)
    \item \textit{points} (optional): A dictionary of points on which to base the analysis (only if the attribute has been changed at this step)
    \item \textit{windows} (optional): A dictionary of windows on which to base the analysis (only if the attribute has been changed at this step)
\end{itemize}

Example:
\begin{lstlisting}
self._history = [
    {
        "function": "__init__",
        "x": np.array([0.0, 1.0, 2.0, 3.0, 4.0, 5.0, 6.0, 7.0, 8.0, 9.0]),
        "y": np.array([0.0, 1.0, 4.0, 2.0, 3.0, 3.0, 5.0, 3.0, 2.0, 1.0]),
        "points": {},
        "windows": {}
    },
    {
        "function": "normalize",
        "y": np.array([0.0, 0.2, 0.8, 0.4, 0.6, 0.6, 1.0, 0.6, 0.4, 0.2])
    }
    {
        "function": "get_point_window",
        "points": {"anti-stokes\_1": 5.5542},
        "windows": {"anti-stokes\_1": (3.0, 8.0)},
    }
]
\end{lstlisting}
