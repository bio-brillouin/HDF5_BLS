The first and main goal of the treatment module is to unify the way users treat their data. This entails an apparent paradox in the way the module has to work. On one hand, the module has to be flexible enough to allow all the data to be treated in a similar way. On the other hand, the module has to be enforce a repeatable way of treating the data. 

To achive these two goals, we ahve decided to think the module as a modulable algorithm maker. This means that the module does not give the user a single function to treat the data. Instead, algorithms combining the standard treatment functions are assembled to create a flexible yet repeatable way of treating the data. The core of the module is therefore to allow the user to either define an algorithm or to apply an existing algorithm to the data.

Defining algorithms can be a tidious task, especially when thousands of data points have to be treated. Adjusting the parameters of the treatment to ensure optimal treatment is also difficult, particularly when only part of the data are not well fitted. To address these issues, the module has been designed to allow the user to re-fit the data with the same algorithm, only with slightly different initial parameters, on the spectra that resulted in errors. Using the treat module is therefore entering in the logic represented by the following flowchart:
\begin{figure}[H]
    \centering
    \label{fig:treatment.flowchart}
    \small
    \begin{tikzpicture}[node distance=2cm]
        \node (start) [startstop, align = center] {Select initial parameters\\ on sampled array};
        \node (run) [process, align = center, right of=start, xshift = 4cm] {Run treatment on dataset\\ with initial parameters};
        \node (errors) [process, align = center, below of=run, xshift=2cm] {Identify spectra that\\ resulted in errors};
        \node (change_parameters) [process, align = center, left of=errors, xshift = -2cm] {Adjust treatment};
        \node (end) [startstop, right of=run, xshift = 4cm] {Return the results};

        \draw [arrow] (start) -- (run);
        \draw [arrow] (run) -- node[anchor=west] {Errors} (errors);
        \draw [arrow] (errors) -- (change_parameters);
        \draw [arrow] (change_parameters) -- (run);
        \draw [arrow] (run) -- node[anchor=north] {No more errors} (end);

    \end{tikzpicture}
\end{figure}

In most cases, a "treatment" will not fundamentally change between a run performed on a "good" spectrum and a run performed on a "bad" spectrum that results in an error. It is therefore interesting to allow the user to re-use the steps that were performed during the initial treatment. To allow this treatment of work on the errors, we can just change the parameters of the function, essentially modifying the initial parameters of the treatment. 

We have therefore developped a silent background utility that stores the steps of the treatment when running functions. Before going into the details of how to re-use a treatment, let's therefore first see how we can use the functions.

