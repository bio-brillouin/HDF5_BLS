The \textit{Treat\_backend} class is the base class for the treat class. Its purpose is to provide the background functions to record, open, create and save algorithms, and to store the different steps of the treatment and their effects on the data. 

The philosophy of this class is to be a silent spie and handyman to the \textit{Treat} class where the treatment codes are stored. The first job of this class is to act as an intermediary between the calling of the function and their execution so as to store what was called and the arguments that were given to the function. This class also acts as a proxy for the user when a standard algorithm has been selected, by asking for the execution of the standard algorithm. This class has the final job of handling these algorithms, presenting them as JSON files, saving them to external files, displaying them properly, etc. This class is not meant to be directly used by the end user and thus modified by most users.

Therefore we highly recommend that before a change is applied on any of the functions of this class  you contact the original developpers of the library to discuss the changes you want to make (if done incorrectly you could break the library).

This class is composed of the following functions:
\begin{itemize}
    \item \textbf{\_\_init\_\_} \\ This function initializes the class. This includes both the initialization of the algorithm arguments and the arguments that will store the data, the treatment-related arguments (initial parameters, points of interest, etc.) and the treatment results. 
    \item \textbf{\_\_getattribute\_\_} \\ This is a redefinition of the native \_\_getattribute\_\_ function of the class. This redefinition is used to store each calling of a function of the class in the \_algorithm attribute of the class together with the arguments that were given to the function. This allows the user to easily store the treatment but also reapply it to the data if needed. Note that this function only records calling to functions whose name doesn't start with an underscore. This function also stores the evolutions of the attributes of the class in the \_history attribute if the \_save\_history attribute is set to True. This is particularly usefull if the user wants to go through the steps of the treatment to verify they are correct without having to re-run all the algorithm.
    \item \textbf{\_clear\_points} \\ This function clears the list of points and the list of windows stored in the class.
    \item \textbf{\_create\_algorithm} \\ This function allows the user to specify the name, version, author and description of the algorithm. It also allows the user to create a new algorithm or overwrite an existing one.
    \item \textbf{\_move\_step} \\ This function allows the user to move a step from one position to another in the \_algorithm attribute. This function deletes the elements of the \_history attribute that are after the moved step (included).
    \item \textbf{\_open\_algorithm} \\ This function allows the user to open an existing JSON algorithm and store it in the \_algorithm attribute. This function also creates an empty history.
    \item \textbf{\_remove\_step} \\ This function allows the user to remove a step from the \_algorithm attribute. This function also deletes the elements of the \_history attribute that are after the removed step (included).
    \item \textbf{\_return\_string\_algorithm} \\ This function returns a string representation of the algorithm stored in the \_algorithm attribute of the class.
    \item \textbf{\_run\_algorithm} \\ This function allows the user to run the algorithm of the class. It also allows the user to run other algorithms if specified. The function runs the algorithm up to the given step (included). If no step is given, the algorithm is run up to the last step (included). 
    \item \textbf{\_save\_algorithm} \\ This function allows the user to save the algorithm to an external JSON file with or without the parameters used. If the parameters are not saved, their value is set to a default value proper to their type.
\end{itemize}

The \textit{Treat\_backend} class is meant to be a low-level class that is never used directly by the user except when the user wants to develop his own treatment platform. In most cases, the user will use the \textit{Treat} class instead, a class that inherits from the \textit{Treat\_backend} class and that is meant to allow the user to define the functions he needs to perform the treatment and to apply them to the data seemlessly. 
