The HDF5\_BLS library is a Python package meant to interface Python code with a HDF5 file with the structure defined in \hyperref[chap:preamble]{Preamble}.

The goal of this package is to allow the user to semalessly integrate the proposed standard to their existing code. A detailed description of the package will be given in the later sections of this tutorial. Here is however a quick code example to show the integration of the package in a simple case:

\begin{lstlisting}
###############################################################################
# Existing imports
###############################################################################
from HDF5_BLS import wrapper

# Create a new file
wrp = wrapper.Wrapper(filepath = "path/to/the/file.h5")

###############################################################################
# Existing code extracting data from a file
###############################################################################

# Store the data in the file 
wrp.add_raw_data(data = data, parent_group = "Brillouin/path/in/the/file", name = "Name of the dataset")

###############################################################################
# Existing code extracting a PSD and a frequency vector from the data
###############################################################################

# Store the frequency vector together with the raw data
wrp.add_frequency(data = frequnecy, parent_group = "Brillouin/path/in/the/file", name = "Frequency vector")

# Store the PSD dataset together with the raw data
wrp.add_PSD(data = PSD, parent_group = "Brillouin/path/in/the/file", name = "PSD")

###############################################################################
# Existing code extracting the shift and linewidth of the data
###############################################################################

# Store the PSD dataset together with the raw data
wrp.add_treated_data(shift = shift, linewidth = linewidth, parent_group = "Brillouin/path/in/the/file", name = "PSD")
\end{lstlisting}

This package also aims at unifying both the way to extract PSD from raw data and extract Brillouin shift and linewidth from the PSD. We will describe later how to do this, we encourage interested readers to already try and add the above code to their code and see how it works. 