The treat module is the module that allows the user to extract information from a Power Spectrum Density. The module can work with all PSD relying on a 1D frequency vector and where the frequency dependence is the last dimension of the PSD. The module allows a seemless storage of the functions that have been called during a treatment. This allows the user to easily store the treatment but also reapply it to the data if needed. This chapter will present the way the module works then the way it can be used and finally how the module can be extended.

The philosophy of the treat module is to apply functions from the class to treat the data. This choice allows for the class to store any of the algorithm steps that have been applied to the data. 

\section{The treatment module}
    Opening the "treat" module you can see 4 different classes:
\begin{itemize}
    \item \textit{Treat\_error}
    \item \textit{Models}
    \item \textit{Treat\_backend}
    \item \textit{Treat}
\end{itemize}

Each of these classes has a specific purpose that will be detailed in the following sections.

\subsection{The \textit{Treat\_error} class}
    The \textit{Treat\_error} class is a class that is used to store the errors that can occur during the treatment. It is the default error type when the treatment fails. This class simply raises an exception for now but can be extended to return the type of error, the position of the point that has caused the error in a dataset, or any other relevant information.

\subsection{The \textit{Models} class}
    The \textit{Models} class is a class that is used to store the analytical models that can be used to fit the data. These can be both fundamental lineshapes (lorentzian, Gaussian, DHO, Voigt, ...) or custom lineshapes for example if a first order Taylor expansion is used to reduce the effect of the elastic peak on the Brillouin scattered peak. 

The class is composed of a series of functions that are used to define the models. These functions are then stored as elements of the \textit{Models.models} dictionary attribute. The keys of this dictionary are the names of the models and the values are the functions that define the model. This choice was made to allow the user to easily select a model by simply writing its name. 

Note that all models have by default an "IR" parameter. This parameter is meant to be the lineshape of the response of the instrument. If this parameter is given, then the model is numerically convolved with the IR. This allows the user to essentially deconvolve the response of the instrument during the fitting of the data.

\subsubsection{Signature}

For now we have restricted the models to functions that can describe a peak by a central position, a width and a height. The signature of functions where no elastic peak correction is applied is tehrefore the following:
\begin{lstlisting}
def function_name(self, nu, b, a, nu0, gamma, IR = None):
    ...
\end{lstlisting}

where:
\begin{itemize}
    \item \textit{self} is the class itself
    \item \textit{nu} is the frequency array
    \item \textit{b} is the constant offset of the data
    \item \textit{a} is the amplitude of the peak
    \item \textit{nu0} is the center position of the function
    \item \textit{gamma} is the linewidth of the function
    \item \textit{IR} is the impulse response of the instrument, by default None
\end{itemize}

The signature of functions where the elastic peak correction is applied by using a first order Taylor expansion is the following:
\begin{lstlisting}
def function_name(self, nu, ae, be, a, nu0, gamma, IR = None:
    ...
\end{lstlisting}

where:
\begin{itemize}
    \item \textit{self} is the class itself
    \item \textit{nu} is the frequency array
    \item \textit{ae} is the slope of the first order Taylor expansion of the elastic peak at the position of the peak fitted
    \item \textit{be} is the constant offset of the data
    \item \textit{a} is the amplitude of the peak
    \item \textit{nu0} is the center position of the function
    \item \textit{gamma} is the linewidth of the function        
    \item \textit{IR} is the impulse response of the instrument, by default None
\end{itemize}

To maintain a seemless compatibility with all the models, we ask the user to use the following convention:
\begin{itemize}
    \item The parameter \textit{gamma} is always the full width at half maximum of the peak. Make sure that your model is defined in this way.
    \item The models are normalized to 1. The parameter \textit{a} is therefore the amplitude of the peak.
    \item The models have no offset. The parameter \textit{b} (or \textit{be} when elastic correction is applied) is therefore the constant offset of the data.
\end{itemize}

\subsubsection{Adding a new model}

To add a new model, you need to 
\begin{itemize}
    \item Create a new function that follows the signature of the functions already present in the class
    \item Add the function to the \textit{Models.models} dictionary attribute
\end{itemize}

For example if you want to add a Gaussian model, you would need to create the following function:
\begin{lstlisting}
def gaussian(self, nu, b, a, nu0, gamma, IR = None):
    ...
\end{lstlisting}

Then you would need to add the function to the \textit{Models.models} dictionary attribute:
\begin{lstlisting}
Models.models["Gaussian"] = lambda nu, b, a, nu0, gamma, IR=None: self.gaussian(nu, b, a, nu0, gamma, IR)
\end{lstlisting}

We strongly encourage the user to add a docstring to the function. As for the rest of the library, numpy style docstrings are used to document the functions. The docstrings should be written in the following way:

\begin{lstlisting}
def function_name(self, nu, b, a, nu0, gamma, IR = None):
    """
    Brief description of the function

    Parameters
    ----------
    nu : array
        The frequency array
    b : float
        The constant offset of the data
    a : float
        The amplitude of the peak
    nu0 : float
        The center position of the function
    gamma : float
        The linewidth of the function
    IR : array, optional
        The impulse response of the instrument, by default None

    Returns
    -------
    function
        The function associated to the given parameters
    """
\end{lstlisting}


\subsection{The \textit{Treat\_backend} class}
    The \textit{Treat\_backend} class is the base class for the treat class. Its purpose is to provide the background functions to record, open, create and save algorithms, and to store the different steps of the treatment and their effects on the data. 

The philosophy of this class is to be a silent spie and handyman to the \textit{Treat} class where the treatment codes are stored. The first job of this class is to act as an intermediary between the calling of the function and their execution so as to store what was called and the arguments that were given to the function. This class also acts as a proxy for the user when a standard algorithm has been selected, by asking for the execution of the standard algorithm. This class has the final job of handling these algorithms, presenting them as JSON files, saving them to external files, displaying them properly, etc. This class is not meant to be directly used by the end user and thus modified by most users.

Therefore we highly recommend that before a change is applied on any of the functions of this class  you contact the original developpers of the library to discuss the changes you want to make (if done incorrectly you could break the library).

This class is composed of the following functions:
\begin{itemize}
    \item \textbf{\_\_init\_\_} \\ This function initializes the class. This includes both the initialization of the algorithm arguments and the arguments that will store the data, the treatment-related arguments (initial parameters, points of interest, etc.) and the treatment results. 
    \item \textbf{\_\_getattribute\_\_} \\ This is a redefinition of the native \_\_getattribute\_\_ function of the class. This redefinition is used to store each calling of a function of the class in the \_algorithm attribute of the class together with the arguments that were given to the function. This allows the user to easily store the treatment but also reapply it to the data if needed. Note that this function only records calling to functions whose name doesn't start with an underscore. This function also stores the evolutions of the attributes of the class in the \_history attribute if the \_save\_history attribute is set to True. This is particularly usefull if the user wants to go through the steps of the treatment to verify they are correct without having to re-run all the algorithm.
    \item \textbf{\_clear\_points} \\ This function clears the list of points and the list of windows stored in the class.
    \item \textbf{\_create\_algorithm} \\ This function allows the user to specify the name, version, author and description of the algorithm. It also allows the user to create a new algorithm or overwrite an existing one.
    \item \textbf{\_move\_step} \\ This function allows the user to move a step from one position to another in the \_algorithm attribute. This function deletes the elements of the \_history attribute that are after the moved step (included).
    \item \textbf{\_open\_algorithm} \\ This function allows the user to open an existing JSON algorithm and store it in the \_algorithm attribute. This function also creates an empty history.
    \item \textbf{\_remove\_step} \\ This function allows the user to remove a step from the \_algorithm attribute. This function also deletes the elements of the \_history attribute that are after the removed step (included).
    \item \textbf{\_return\_string\_algorithm} \\ This function returns a string representation of the algorithm stored in the \_algorithm attribute of the class.
    \item \textbf{\_run\_algorithm} \\ This function allows the user to run the algorithm of the class. It also allows the user to run other algorithms if specified. The function runs the algorithm up to the given step (included). If no step is given, the algorithm is run up to the last step (included). 
    \item \textbf{\_save\_algorithm} \\ This function allows the user to save the algorithm to an external JSON file with or without the parameters used. If the parameters are not saved, their value is set to a default value proper to their type.
\end{itemize}

The \textit{Treat\_backend} class is meant to be a low-level class that is never used directly by the user except when the user wants to develop his own treatment platform. In most cases, the user will use the \textit{Treat} class instead, a class that inherits from the \textit{Treat\_backend} class and that is meant to allow the user to define the functions he needs to perform the treatment and to apply them to the data seemlessly. 


\subsection{The \textit{Treat} class}
    The \textit{Treat} class is the class that is meant to be used by the user to define the functions he needs to perform the treatment and to apply them to the data seemlessly. This class inherits from the \textit{Treat\_backend} class and thus inherits all the functions of the latter.

\subsubsection{Organization of the class}

The \textit{Treat} class is divided in 8 different types of functions:
\begin{itemize}
    \item \textit{Point and window definition}: functions that are used to define the points and windows of interest on the data. These can then be used to apply correction to the data (like a normalization) or a fitting.
    \item \textit{Estimation functions}: functions that are used to estimate the parameters of the model.
    \item \textit{Fitting model definition}: functions that are used to define the model to be used to fit the data.
    \item \textit{Automatic estimation functions}: functions that are used to automatically estimate the parameters of the model.
    \item \textit{Fitting functions}: functions that are used to fit the data.
    \item \textit{Post-treatment functions}: functions that are used to perform post-treatment tasks on the data.
    \item \textit{Outliers and errors functions}: functions that are used to identify outliers and errors in the data.
    \item \textit{Algorithm application functions}: functions that are used to apply the algorithm to the data.
\end{itemize}

\subsubsection{Return of functions}

All the functions of the \textit{Treat} class are silent by default, meaning that the result of the function is not returned and they only modify the attributes of the class. This strategy allows for a reliable tracking of the treatment, its effect on the data and the ability to examine the treatment at any step of the algorithm.

The attributes of the class that are modified by the functions are:
\begin{itemize}
    \item \textit{\_treat\_selection}: a string that directs the treatment towards the whole PSD array, sampled PSD arrays or error points with the following options:
        \begin{itemize}
            \item \textit{all}: the whole PSD array is treated
            \item \textit{sampled}: only the PSD array stored in the \textit{PSD\_sample} attribute is treated
            \item \textit{errors}: only the spectra in the PSD dataset whose position are stored in the \textit{point\_error} attribute are treated
        \end{itemize}
    \item \textit{point\_error}: the position of the spectra that lead to an error, in the PSD dataset. This is a list of tuples.
    \item \textit{point\_error\_type}: the type of the error (fit error, shift outlier, etc.) corresponding to each point in the \textit{point\_error} attribute. This is a list of strings.
    \item \textit{point\_error\_value}: the value of the error (NaN if fit error, the shift value if shift outlier, etc.) corresponding to each point in the \textit{point\_error} attribute. This is a list of floats.
    \item \textit{PSD\_sample}: the PSD array on which the treatment is performed. This can either be an average of the whole PSD on the frequency axis or any individual PSD of the array.
    \item \textit{frequency\_sample}: the frequency array on which the treatment is performed. This is equivalent to the \textit{frequency} attribute of the class for 1D frequency arrays (the only available option for now).
    \item \textit{points}: a list of points that are not well fitted
    \item \textit{windows}: a list of windows that are not well fitted
    \item \textit{width\_estimator}: the estimation of the width of each peak. This is a list of floats with the same length as the \textit{points} attribute.
    \item \textit{fit\_model}: the string corresponding to the model used to fit the data, matching one of the keys of the \textit{Models.models} dictionnary.
    \item \textit{shift}: the shift array obtained after the treatment
    \item \textit{linewidth}: the linewidth array obtained after the treatment
    \item \textit{shift\_var}: the array of standard deviation of the shift array obtained after the treatment
    \item \textit{linewidth\_var}: the array of standard deviation of the linewidth array obtained after the treatment
    \item \textit{amplitude}: the amplitude array obtained after the treatment
    \item \textit{amplitude\_var}: the array of standard deviation of the amplitude array obtained after the treatment
    \item \textit{BLT}: the BLT array obtained after the treatment
    \item \textit{BLT\_var}: the array of standard deviation of the BLT array obtained after the treatment
\end{itemize}

\subsubsection{Functions signature}

Each function defined in the \textit{Treat} class uses key-word only arguments where the type of the argument is specified in the function signature. Defining the type of the arguments allows for an easier integration of the treatment in the GUI. The arguments with type \textit{booleans} for example will be recognized and presented to the user as a checkbox in the GUI.

Using key-word only arguments also ensures that the function can be called without any argument without raising an error. This is particularly usefull to create algorithms without parameters by just calling the functions one after the other. Note that for a safe-by-default definition of the functions, we recommend defining a specific breakpoint at the beginning of the function to check if arguments are given to the function and return nothing if they are not.

To further improve the seamless integration of new functions to the GUI, arguments of the function have a preferred nomenclature. This nomenclature works by using defined prefixes to the arguments to orient the best way to define them in the GUI. The list of prefixes is extendable. In the actual version of the library, these prefixes are:
\begin{itemize}
    \item \textit{position}: The argument refers to a position on the x axis of the data. The GUI recognizes this parameter and allows the user to click on the graph to select a position.
    \item \textit{window}: The width of the window around the peak. The GUI recognizes this parameter and creates a text box to enter the value of the window.
    \item \textit{type\_peak}: The type of the peak. Can be either "Stokes", "Anti-Stokes" or "Elastic".
    \item \textit{center\_type}: The type of the peak to center the x axis around. Must be either "Elastic" or "Inelastic".
    \item \textit{model}: The model to be used to fit the data. The GUI recognizes this parameter and creates a combobox to select the model based on the keys of the \textit{Models.models} dictionnary.
\end{itemize}

All other arguments will be displayed in the GUI as text boxes. 

Note that using the beforementionned nomenclature of the arguments is not mandatory for the function to work.

\subsubsection{Example of a function}

Let's define here a function that applies a Savitsky-Golay filter to the data before fitting. This function is meant to be used before the fitting so we'll define it as a function that is called before the fitting. A Savitsky-Golay filter is a type of filter that is used to smooth the data by fitting a polynomial to the data. The function will be applied on the classe's argument "PSD" takes the following arguments:
\begin{itemize}
    \item \textit{window}: The length of the window to be used to fit the data.
    \item \textit{order}: The order of the polynomial to be used to fit the data.
\end{itemize}

Both of these parameters are integers so we don't need to use a particular prefix for them. We'll give the function initial values to be recognized as the break case. The signature of the function is therefore the following:

\begin{lstlisting}
def savgol_filter(self, window : int = None, order : int = None):
    ...
\end{lstlisting}

We can now define the break case of the function when called without arguments:
\begin{lstlisting}
def savgol_filter(self, window = None, order = None):
    if window is None or order is None:
        return
    ...
\end{lstlisting}

Following this step, we can just apply the function to the data. Note that fits are applied to the attribute "PSD\_sample", and attribute that is updated when calling "apply\_algorithm\_on\_all". Therefore here, the function will write:

\begin{lstlisting}
def savgol_filter(self, window = None, order = None):
    if window is None or order is None:
        return
    self.PSD_sample = signal.savgol_filter(self.PSD_sample, window_length = window, polyorder = order)
\end{lstlisting}
 
The last step here is to add the docstring of the function. Here we could for example add the following:
\begin{lstlisting}
def savgol_filter(self, window = None, order = None):
    """
    Applies a Savitsky-Golay filter to the given PSD.

    Parameters
    ----------
    window : int, optional
        The length of the window to be used to fit the data, by default None.
    order : int, optional
        The order of the polynomial to be used to fit the data, by default None.
    """
    if window is None or order is None:
        return
    self.PSD_sample = signal.savgol_filter(self.PSD_sample, window_length = window, polyorder = order)
\end{lstlisting}


\section{Using the treatment module}
    The treatment module has the particularity of being able to be used in four different ways:
\begin{itemize}
    \item To simply use the functions it provides to treat the data
    \item To define a new treatment algorithm
    \item To apply an existing treatment algorithm to the data
    \item To re-use an existing treatment and optimize it
\end{itemize}

In the following sections, we will present the different ways to use the treatment module.

\subsection{I just have raw data coming from the spectrometer}

\section{Extending the treatment module}
    \input{tex/01-Tutorial/1-HDF5_BLS/5-Treat/3-Extensions.tex}