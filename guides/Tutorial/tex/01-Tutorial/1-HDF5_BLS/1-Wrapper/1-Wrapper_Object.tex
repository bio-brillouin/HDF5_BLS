The "wrapper" module has one main object: \textit{Wrapper}. This object is used to interact with the HDF5 file. It is used to read the data, to write the data and to modify any aspect of the HDF5 file (dataset, groups or attributes). The module also provides different error objects used to recognize errors when using the Wrapper object and raise exceptions.

The Wrapper object is initialized by running the following command:
\begin{lstlisting}
wrp = Wrapper()
\end{lstlisting}

This will create a new Wrapper object with no attributes or data, and with the following structure:
\begin{verbatim}
    file.h5
    +-- Brillouin (group)
\end{verbatim}

By default, the attributes of the "Brillouin" group are the following:
\begin{verbatim}
    file.h5
    +-- Brillouin (group)
    |   +-- Brillouin_type -> "Root"
    |   +-- HDF5_BLS_version -> "0.1" 
\end{verbatim}

As long as no filepaths are given to the Wrapper object, the file is stored in the library folder. Note that this temporary file is deleted either when the Wrapper object is destroyed or when the file is stored elsewhere. It is therefore good practice to specify a non-temporary filepath to the file when creating a new Wrapper object, with the "filepath" parameter:
\begin{lstlisting}
wrp = Wrapper(filepath = "path/to/file.h5")
\end{lstlisting}
Note that this works both for new files, and for files that already exist, in the latter case, the wrapper object applies to the file located at "path/to/file.h5".
