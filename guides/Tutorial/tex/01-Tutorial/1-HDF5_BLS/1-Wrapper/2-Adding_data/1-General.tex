
Adding a dataset to the file always come with three other pieces of information:
\begin{itemize}
    \item Where to add the dataset in the file
    \item What to call the added dataset
    \item What is the type of the dataset we want to add
\end{itemize}

To add a dataset to the file, we'll therefore call type-specific functions with the data to add, the place where to add it and the name to give the dataset as arguments, following:

\begin{figure}[H]
    \centering
    \label{fig:wrapper.flowchart_add_functions}
    \small
    \begin{tikzpicture}[node distance=2cm]
        \node (start) [startstop] {\textit{data} is a dataset to add};
        \node (parentGroup) [startstop, right of=start, yshift=0cm, xshift=3cm, align=center] {\textit{parent\_group} is the parent group \\ where to store the \\data in the HDF5 file};
        \node (name) [startstop, right of=parentGroup, yshift=0cm, xshift=3cm, align=center] {\textit{name} is the name of the \\dataset to add};
        \node (function) [process, below of=parentGroup, yshift=0cm, xshift=0cm, align=center] {\textit{type-specific\_function}(data = \textit{data}, \\parent\_group = \textit{parent\_group}, \\name = \textit{name})};

        \draw [arrow] (start) -- (function);
        \draw [arrow] (parentGroup) -- (function);
        \draw [arrow] (name) -- (function);
    \end{tikzpicture}
    \caption{General approach for adding data to the HDF5 file}
\end{figure}

This approach is the one used for
\begin{itemize}
    \item \hyperref[subchapter:wrapper.add_raw_data]{add\_raw\_data}
    \item \hyperref[subchapter:wrapper.add_psd]{add\_PSD}
    \item \hyperref[subchapter:wrapper.add_frequency]{add\_frequency}
    \item \hyperref[subchapter:wrapper.add_other]{add\_other}
\end{itemize}

\begin{center}
    \rule{15cm}{0.4pt}
\end{center}

\paragraph*{Example}
Let's consider the following example: we have just initialized a wrapper object and want to add a spectrum obtained from our spectrometer. We have already converted this spectrum to a numpy array, and named it \textit{data}. Now we want to add this data in a group called "Water spectrum" in the root group of the HDF5 file and call this raw data "Measure of the year". Then we will write:
\begin{lstlisting}
wrp.add_raw_data(data = data,
                 parent_group = "Brillouin/Water spectrum", 
                 name = "Measure of the year")
\end{lstlisting}

Now let's say that we have analyzed this spectrum and obtained a PSD (stored in the variable "psd") and frequency array (stored in the variable "freq"). We want to add these two arrays in the same group, and call them "PSD" and "Frequency" respectively. We will write:
\begin{lstlisting}
wrp.add_PSD(data = psd,
            parent_group = "Brillouin/Water spectrum", 
            name = "PSD")
wrp.add_frequency(freq,
                  parent_group = "Brillouin/Water spectrum", 
                  name = "Frequency")
\end{lstlisting}

\begin{center}
    \rule{15cm}{0.4pt}
\end{center}