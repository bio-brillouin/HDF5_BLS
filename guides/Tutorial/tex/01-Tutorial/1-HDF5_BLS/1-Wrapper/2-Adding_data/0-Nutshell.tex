% \begin{tcolorbox}
% \textit{IN A NUTSHELL:}
% \begin{itemize}
%     \item With the \hyperref[subchapter:wrapper.add_dictionnary]{add\_dictionnary} method 
% \begin{lstlisting}
% wrp = Wrapper()
% dic = {"Raw_data": {"Name": "Wonderful measure 019", 
%                     "Data": np.random.random((100, 100))},
%        "PSD": {"Name": "PSD extracted blabla", 
%                "Data": np.random.random((100, 100))},
%        "Frequency": {"Name": "The Frequency",   
%                    "Data": np.random.random((100))},
%        "Abscissa_t": {"Name": "Time (s)", 
%                       "Data": np.random.random((100)),
%                       "Unit": "s",
%                       "Dim_start": "0",
%                       "Dim_end": "1"}}
% wrp.add_dictionnary(dic, 
%                     parent_group = "Brillouin", 
%                     name = "Data_0", 
%                     brillouin_type = "Measure",
%                     overwrite = False)
% \end{lstlisting}
%     \item With specific methods for each type of data:
%     \begin{itemize}
%         \item \hyperref[subchapter:wrapper.add_raw_data]{add\_raw\_data}: To add raw data to a group
% \begin{lstlisting}
% wrp = Wrapper()
% data = np.random.random((10, 10, 512)) # The raw data that you want to add
% wrp.add_raw_data(data,
%                 parent_group = "Brillouin/Data_0", 
%                 name = "Raw_data")
% \end{lstlisting}
%         \item \hyperref[subchapter:wrapper.add_psd]{add\_PSD}: To add a PSD to a group
% \begin{lstlisting}
% wrp = Wrapper()
% data = np.random.random((10, 10, 512)) # The PSD that you want to add
% wrp.add_PSD(data,
%             parent_group = "Brillouin/Data_0", 
%             name = "PSD")
% \end{lstlisting}
        
% \end{itemize}
% \end{itemize}
% \end{tcolorbox}
% \begin{tcolorbox}
% \begin{itemize}
%     \item[] \begin{itemize}
%         \item \hyperref[subchapter:wrapper.add_frequency]{add\_frequency}: To add a frequency axis to a group
% \begin{lstlisting}
% wrp = Wrapper()
% data = np.random.random((512)) # The frequency axis that you want to add
% wrp.add_frequency(data,
%                   parent_group = "Brillouin/Data_0", 
%                   name = "Frequency")
% \end{lstlisting}
%         \item \hyperref[subchapter:wrapper.add_abscissa]{add\_abscissa}: To add an abscissa to a group
% \begin{lstlisting}
% wrp = Wrapper()
% data = np.random.random((10, 10)) # The abscissa that you want to add
% wrp.add_abscissa(data,
%                  parent_group = "Brillouin/Data_0", 
%                  name = "x and y",
%                  unit = "microns",
%                  dimension_PSD_start = 0,
%                  dimension_PSD_end = 1)
% \end{lstlisting}
%         \item \hyperref[subchapter:wrapper.add_treated_data]{add\_treated\_data}: To add a shift, linewidth and their respective errors to a dedicated "Treatment" group
% \begin{lstlisting}
% wrp = Wrapper()
% shift = np.random.random((10, 10)) # The shift array to add
% linewidth = np.random.random((10, 10)) # The linewidth array to add
% shift_err = np.random.random((10, 10)) # The shift error array to add
% linewidth_err = np.random.random((10, 10)) # The linewidth error array to add
% wrp.add_treated_data(shift = shift,
%                      linewidth = linewidth,
%                      shift_err = shift_err,
%                      linewidth_err = linewidth_err,
%                      parent_group = "Brillouin/Data_0", 
%                      name_group = "NnMF - 5GHz")
% \end{lstlisting}
%     \end{itemize}
% \end{itemize}
% \end{tcolorbox}

The addition of any type of data or attribute to the HDF5 file has been centralized in the \hyperref[subchapter:wrapper.add_dictionary]{Wrapper.add\_ dictionary} method. This method is safe but complex and not user-friendly. Methods derived from this method are meant to simplify the process of adding data to the HDF5 file, specific to each type of data. To get a better understanding on how to use the add\_dictionary method, please refer to the \hyperref[subchapter:wrapper.add_dictionary]{developper guide section}.

To add a single dataset to a group, we first need to specify the type of dataset we want to add, following the ones presented in \hyperref[subsec:preamble.file_structure.complete_structure]{preamble}:
\begin{itemize}
    \item "Abscissa\_...": An abscissa array for the measures where the dimensions on which the dataset applies are given after the underscore.
    \item "Amplitude": The dataset contains the values of the fitted amplitudes.
    \item "Amplitude\_err": The dataset contains the error of the fitted amplitudes.
    \item "BLT": The dataset contains the values of the fitted amplitudes.
    \item "BLT\_err": The dataset contains the error of the fitted amplitudes.
    \item "Frequency": A frequency array associated to the power spectral density
    \item "Linewidth": The dataset contains the values of the fitted linewidths.
    \item "Linewidth\_err": The dataset contains the error of the fitted linewidths. 
    \item "PSD": A power spectral density array
    \item "Raw\_data": The dataset containing the raw data obtained after a BLS experiment.
    \item "Shift": The dataset contains the values of the fitted frequency shifts.
    \item "Shift\_err": The dataset contains the error of the fitted frequency shifts.
    \item "Other": The dataset contains other data that will not be used by the library.
\end{itemize}

From there, the following functions are available to add the dataset to the HDF5 file:
\begin{itemize}
    \item \hyperref[subchapter:wrapper.add_raw_data]{add\_raw\_data}: To add raw data to a group
    \item \hyperref[subchapter:wrapper.add_psd]{add\_PSD}: To add a PSD to a group
    \item \hyperref[subchapter:wrapper.add_frequency]{add\_frequency}: To add a frequency axis to a group
    \item \hyperref[subchapter:wrapper.add_abscissa]{add\_abscissa}: To add an abscissa to a group
    \item \hyperref[subchapter:wrapper.add_treated_data]{add\_treated\_data}: To add a shift, linewidth and their respective errors to a dedicated "Treatment" group
    \item \hyperref[subchapter:wrapper.add_other]{add\_other}: To add a shift, linewidth and their respective errors to a dedicated "Treatment" group
\end{itemize}