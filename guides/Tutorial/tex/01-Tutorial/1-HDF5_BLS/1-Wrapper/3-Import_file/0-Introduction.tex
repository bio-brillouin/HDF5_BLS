% \begin{tcolorbox}
%     \textit{IN A NUTSHELL:}

%     Importing data from files is done with specific methods for each type of data that is imported:
%     \begin{itemize}
%         \item \hyperref[subchapter:wrapper.import_abscissa]{Wrapper.import\_abscissa}: To import an abscissa from a measure file.
% \begin{lstlisting}
%     wrp = Wrapper()
%     wrp.import_abscissa(filepath = "path/to/file.txt", parent_group = "Brillouin/Measure", creator = None, parameters = None, name = "Time", unit = "s", dim_start = 0, dim_end = 1, reshape = None, overwrite = False)
% \end{lstlisting}
%         \item \hyperref[subchapter:wrapper.import_frequency]{Wrapper.import\_frequency}: To import a frequency array from a measure file.
% \begin{lstlisting}
%     wrp.import_frequency(filepath = "path/to/file.txt", parent_group = "Brillouin/Measure", creator = None, parameters = None, name = "Frequency", reshape = None, overwrite = False)
% \end{lstlisting}
%         \item \hyperref[subchapter:wrapper.import_psd]{Wrapper.import\_PSD}: To import a Power Spectral Density array from a measure file.
% \begin{lstlisting}
%     wrp.import_PSD(filepath = "path/to/file.txt", parent_group = "Brillouin/Measure", creator = None, parameters = None, name = "PSD", reshape = None, overwrite = False)
% \end{lstlisting}
%         \item \hyperref[subchapter:wrapper.import_raw_data]{Wrapper.import\_raw\_data}: To import a Raw data array from a measure file.
% \begin{lstlisting}
%     wrp.import_raw_data(filepath = "path/to/file.txt", parent_group = "Brillouin/Measure", creator = None, parameters = None, name = "PSD", reshape = None, overwrite = False)
% \end{lstlisting}
%         \item \hyperref[subchapter:wrapper.import_treated_data]{Wrapper.import\_treated\_data}: To import the data arrays resulting from a treatment.
% \begin{lstlisting}
%     wrp.import_treated_data(self, filepath_shift, filepath_linewidth, filepath_shift_err, filepath_linewidth_err, parent_group, creator = None, parameters = None, name = None, reshape = None, overwrite = False):
% \end{lstlisting}
%     \end{itemize}

%     \textbf{Note:} Importing the data is done using the load\_data module. All the data format and extensions handled by the module are described in the \hyperref[chap:load_data]{load\_data section}.
% \end{tcolorbox}
    
    
Importing datasets to the HDF5 file from independent data files, through the HDF5\_BLS package, is always done following to successive steps:
\begin{enumerate}
    \item Extracting the data and the metadata that can be extracted from the data files. This is done using the \hyperref[chap:load_data]{load\_data module}.
    \item Adding the data and metadata to the HDF5 file. This is done using the \hyperref[subchapter:wrapper.add_dictionary]{Wrapper.add\_dictionary} method.
\end{enumerate}

To make the process more user friendly, we have developed a set of derived methods that are specific to each type of data that is to be added (Raw data, PSD, Frequency, Abscissa or treated data). 

In this section, we will present these methods. We encourage interested readers to refer to the \hyperref[chap:load_data]{chapter dedicated to the load\_data module} for more information on the extraction of the data and the metadata.