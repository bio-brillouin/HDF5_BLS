\begin{tcolorbox}
    \textit{IN A NUTSHELL:}

    Importing data from files is done with specific methods for each type of data that is imported:
    \begin{itemize}
        \item \hyperref[subsec:wrapper.import_abscissa]{Wrapper.import\_abscissa}: To import an abscissa from a measure file.
\begin{lstlisting}
    wrp = Wrapper()
    wrp.import_abscissa(filepath = "path/to/file.txt", parent_group = "Brillouin/Measure", creator = None, parameters = None, name = "Time", unit = "s", dim_start = 0, dim_end = 1, reshape = None, overwrite = False)
\end{lstlisting}
        \item \hyperref[subsec:wrapper.import_frequency]{Wrapper.import\_frequency}: To import a frequency array from a measure file.
\begin{lstlisting}
    wrp.import_frequency(filepath = "path/to/file.txt", parent_group = "Brillouin/Measure", creator = None, parameters = None, name = "Frequency", reshape = None, overwrite = False)
\end{lstlisting}
        \item \hyperref[subsec:wrapper.import_psd]{Wrapper.import\_PSD}: To import a Power Spectral Density array from a measure file.
\begin{lstlisting}
    wrp.import_PSD(filepath = "path/to/file.txt", parent_group = "Brillouin/Measure", creator = None, parameters = None, name = "PSD", reshape = None, overwrite = False)
\end{lstlisting}
        \item \hyperref[subsec:wrapper.import_raw_data]{Wrapper.import\_raw\_data}: To import a Raw data array from a measure file.
\begin{lstlisting}
    wrp.import_raw_data(filepath = "path/to/file.txt", parent_group = "Brillouin/Measure", creator = None, parameters = None, name = "PSD", reshape = None, overwrite = False)
\end{lstlisting}
        \item \hyperref[subsec:wrapper.import_treated_data]{Wrapper.import\_treated\_data}: To import the data arrays resulting from a treatment.
\begin{lstlisting}
    wrp.import_treated_data(self, filepath_shift, filepath_linewidth, filepath_shift_std, filepath_linewidth_std, parent_group, creator = None, parameters = None, name = None, reshape = None, overwrite = False):
\end{lstlisting}
    \end{itemize}

    \textbf{Note:} Importing the data is done using the load\_data module. All the data format and extensions handled by the module are described in the \hyperref[chap:load_data]{load\_data section}.
\end{tcolorbox}
    
\section{An overview}
    
    Importing datasets to the HDF5 file from independent data files, through the HDF5\_BLS package, is always done following to successive steps:
    \begin{enumerate}
        \item Extracting the data and the metadata that can be extracted from the data files. This is done using the \hyperref[chap:load_data]{load\_data module}.
        \item Adding the data and metadata to the HDF5 file. This is done using the \hyperref[subsec:wrapper.add_dictionnary]{Wrapper.add\_dictionnary} method.
    \end{enumerate}

    To make the process more user friendly, we have developed a set of derived methods that are specific to each type of data that is to be added (Raw data, PSD, Frequency, Abscissa or treated data). 

\section{The logic behind the adding functions}
    Following the same approach as described in the previous chapter, we import data in function of their nature, using dedicated functions. These functions are:
    \begin{itemize}
        \item \hyperref[subsec:wrapper.import_abscissa]{Wrapper.import\_abscissa}: To import an abscissa array.
        \item \hyperref[subsec:wrapper.import_frequency]{Wrapper.import\_frequency}: To import a frequency array.
        \item \hyperref[subsec:wrapper.import_psd]{Wrapper.import\_PSD}: To import a PSD array.
        \item \hyperref[subsec:wrapper.import_raw_data]{Wrapper.import\_raw\_data}: To import raw data.
        \item \hyperref[subsec:wrapper.import_treated_data]{Wrapper.import\_treated\_data}: To import the data arrays resulting from a treatment.
    \end{itemize}

    The logic behind the import functions is the following:
    \begin{figure}[H]
        \centering
        \label{fig:wrapper.flowchart_import_functions}
        \small
        \begin{tikzpicture}[node distance=1cm]
            \node (start) [startstop, align=center] {We want to import the data\\stored in the file \textit{filepath}\\to the HDF5 file};
            \node(isAbscissa) [process, below of=start, yshift=-1cm, xshift=2cm, align=center] {We want to import\\an abscissa array};
            \node(isFreq) [process, below of=isAbscissa, yshift=-0.5cm, xshift=0cm, align=center] {We want to import\\a frequency array};
            \node(isPSD) [process, below of=isFreq, yshift=-0.5cm, xshift=0cm, align=center] {We want to import\\a PSD array};
            \node(israw) [process, below of=isPSD, yshift=-0.5cm, xshift=0cm, align=center] {We want to import\\a raw data array};
            \node(isTreated) [process, below of=israw, yshift=-0.5cm, xshift=0cm, align=center] {We want to import\\the data arrays\\after treatment};
            \node(extAbs) [process, right of=isAbscissa, yshift=0cm, xshift=3.5cm, align=center] {Extract data and metadata\\from the file};
            \node(extFreq) [process, right of=isFreq, yshift=0cm, xshift=3.5cm, align=center] {Extract data and metadata\\from the file};
            \node(extPSD) [process, right of=isPSD, yshift=0cm, xshift=3.5cm, align=center] {Extract data and metadata\\from the file};
            \node(extraw) [process, right of=israw, yshift=0cm, xshift=3.5cm, align=center] {Extract data and metadata\\from the file};
            \node(extTreated) [process, right of=isTreated, yshift=0cm, xshift=3.5cm, align=center] {Extract data and metadata\\from the file};
            \node(fAbs) [process, right of=extAbs, yshift=0cm, xshift=3.5cm, align=center] {Wrapper.add\_abscissa};
            \node(fFreq) [process, right of=extFreq, yshift=0cm, xshift=3.5cm, align=center] {Wrapper.add\_frequency};
            \node(fPSD) [process, right of=extPSD, yshift=0cm, xshift=3.5cm, align=center] {Wrapper.add\_PSD};
            \node(fraw) [process, right of=extraw, yshift=0cm, xshift=3.5cm, align=center] {Wrapper.add\_raw\_data};
            \node(fTreated) [process, right of=extTreated, yshift=0cm, xshift=3.5cm, align=center] {Wrapper.add\_treated\_data};

            \draw[arrow] (start) |- node[anchor=west, yshift=1cm] {Depending on the type of data} (isAbscissa);
            \draw[arrow] (start) |- (isFreq);
            \draw[arrow] (start) |- (isPSD);
            \draw[arrow] (start) |- (israw);
            \draw[arrow] (start) |- (isTreated);
            \draw[arrow] (isAbscissa) -- (extAbs);
            \draw[arrow] (isFreq) -- (extFreq);
            \draw[arrow] (isPSD) -- (extPSD);
            \draw[arrow] (israw) -- (extraw);
            \draw[arrow] (isTreated) -- (extTreated);
            \draw[arrow] (extAbs) -- (fAbs);
            \draw[arrow] (extFreq) -- (fFreq);
            \draw[arrow] (extPSD) -- (fPSD);
            \draw[arrow] (extraw) -- (fraw);
            \draw[arrow] (extTreated) -- (fTreated);    
        \end{tikzpicture}
        \caption{Flowchart of the import functions}
    \end{figure}

\section{The arguments of the import functions}
    The import functions combine both the arguments needed for the "add\_" functions and the arguments needed to extract the data and metadata from the files. These are the following arguments:
    \begin{itemize}
        \item \textbf{filepath}: The path to the file containing the data to import. Note that for the "Wrapper.import\_ treated\_data" function, there are 4 different files to import (for the fshift, the linewidth, the shift\_std and the linewidth\_std arrays).
        \item \textbf{parent\_group}: The path to the group or dataset where the data will be added in the form "Brillouin/Measure/...".
        \item \textbf{creator} \textit{(optional, default None)}: Some file extensions are not structured in the same way. Depending on how particular labs store their data, the "creator" argument can be used to specify the structure of the file that has to be loaded. In most cases, this argument is not used and can be left to None. If it however has to be used, a LoadError\_creator will be raised.
        \item \textbf{parameters} \textit{(optional, default None)}: The parameters that are to be used to import the data correctly. These parameters are specific to the techniques used to obtain the data. In most cases, this argument is not used and can be left to None. If it however has to be used, a LoadError\_parameters will be raised.
        \item \textbf{name} \textit{(optional, default None)}: The name that will be given to the dataset. If None, the name is set to whatever the type of data is (e.g. "Frequency").
        \item \textbf{reshape} \textit{(optional, default None)}: The new shape of the array. If None, the shape is not changed.    
        \item \textbf{overwrite} \textit{(optional, default False)}: If True, the attributes of the selected group or dataset are overwritten if they already exist in the file.
    \end{itemize}
