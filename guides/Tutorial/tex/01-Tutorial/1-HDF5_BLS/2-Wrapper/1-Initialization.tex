\begin{tcolorbox}
\textit{IN A NUTSHELL:}
\begin{lstlisting}
wrp = Wrapper(filepath = "path/to/file.h5")
\end{lstlisting}
"path/to/file.h5" is the path where you want to store the file or where the file is stored.
\end{tcolorbox}


To initialize a Wrapper object after having imported the library, you can run the following command:
\begin{lstlisting}
wrp = Wrapper()
\end{lstlisting}

This will create a new Wrapper object with no attributes or data with the following structure:
\begin{verbatim}
    file.h5
    +-- Brillouin (group)
\end{verbatim}

By default, the attributes of the "Brillouin" group are the following:
\begin{verbatim}
    file.h5
    +-- Brillouin (group)
    |   +-- Brillouin_type -> "Root"
    |   +-- HDF5_BLS_version -> "0.1" 
\end{verbatim}

This file is stored temporarily in the library folder and is deleted when the Wrapper object is destroyed or when the file is stored elsewhere. It is therefore good practice to specify a non-temporary filepath to the file when creating a new Wrapper object, with the "filepath" parameter:
\begin{lstlisting}
wrp = Wrapper(filepath = "path/to/file.h5")
\end{lstlisting}
Note that this works both for new files, and for files that already exist, in the latter case, the wrapper object applies to the file located at "path/to/file.h5".
