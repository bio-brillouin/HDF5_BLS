If you already have a script dedicated to the opening, reading, treating and saving of your data, you can use the package to just create a HDF5 file where all your data will be stored. This is the easiest way to 

the most straightforward way to use the package, and it is the one we recommend to use when you are not familiar with the package, as it is meant to be fast, easy and simple to do.


you can save what you already have in the proposed file format. This option is the most straightforward one, and it is the one we recommend to use when you are not familiar with the package, as it is meant to be fast, easy and simple to do.

Usually, you find yourself with the following datasets:
\begin{itemize}
    \item The raw array that came from your spectrometer
    \item A power spectral density array with its associated frequency axis
    \item Some extracted information from the raw array (e.g. shift, linewidth, amplitude, etc.)
\end{itemize}

One idea would be to save these datasets in the following structure:
\begin{verbatim}
    file.h5
    +-- Brillouin (group)
    |   +-- Measure (group)
    |   |   +-- Raw data (dataset)
    |   |   +-- PSD (dataset)
    |   |   +-- Frequency (dataset)
    |   |   +-- Results of fit (group)
    |   |   |   +-- Shift (dataset)
    |   |   |   +-- Linewidth (dataset)
    |   |   |   +-- Amplitude (dataset)
\end{verbatim}


To create this structure, you can add these lines to your script:
\begin{lstlisting}
###############################################################################
# Existing code importing libraries
###############################################################################
from HDF5_BLS import Wrapper

wrp = Wrapper(filepath = "path/to/file.h5") # Replace with the path to your file

###############################################################################
# Existing code extracting the raw data
###############################################################################
# Add the raw data
wrp.add_raw_data(data = raw_data, parent_group = "Brillouin/Measure", name = "Raw data")

###############################################################################
# Existing code converting the raw data into a PSD and frequency axis
###############################################################################
# Add the PSD
wrp.add_PSD(data = psd, parent_group = "Brillouin/Measure", name = "PSD")

# Add the frequency axis
wrp.add_frequency(data = frequency, parent_group = "Brillouin/Measure", name = "Frequency")

###############################################################################
# Existing code extracting shift, linewidth, amplitude
###############################################################################
# Add the shift, linewidth, amplitude, etc.
wrp.add_treated_data(shift = shift, linewidth = linewidth, amplitude = amplitude, parent_group = "Brillouin/Measure/", name_group = "Results of fit")
\end{lstlisting}

To access the datasets in the resulting file, you can then use the following code:
\begin{lstlisting}
from HDF5_BLS import Wrapper

wrp = Wrapper(filepath = "path/to/file.h5") # Replace with the path to your file

raw_data = wrp["Brillouin/Measure/Raw data"]
psd = wrp["Brillouin/Measure/PSD"]
frequency = wrp["Brillouin/Measure/Frequency"]
shift = wrp["Brillouin/Measure/Results of fit/Shift"]
linewidth = wrp["Brillouin/Measure/Results of fit/Linewidth"]
amplitude = wrp["Brillouin/Measure/Results of fit/Amplitude"]
\end{lstlisting}

