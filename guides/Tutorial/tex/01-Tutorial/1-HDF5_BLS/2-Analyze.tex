The "analyze" module is meant to allow users relying on any spectrometer to convert the data they have extracted using their instrument to a physically meaningful format: a couple of datasets storing the Power Spectral Density and the frequency axis. 

This module is built around a hierarchical structure of classes with a mother class meant to be the backbone of the module and children classes meant to be instrument-specific.

Using the analyze module essentially consists in the following steps:
\begin{enumerate}
    \item The user calls the class corresponding to the type of spectrometer they have
    \item The user either opens an existing algorithm to apply to the data or creates a new one
    \begin{enumerate}
        \item If an algorithm is opened, the user adapts the parameters of the algorithm to their data
    \end{enumerate}
    \item The user runs the code and obtains a Power Spectral Density and a frequency axis
\end{enumerate}

