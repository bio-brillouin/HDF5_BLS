In this seventh example, we will store a single measure where two different treatments have been performed (for example a measure at an interface between two materials).

The following structure represents the base structure of the file:
\begin{verbatim}
    file.h5
    +-- Data (group) -> Name = "Measure"
    |   +-- Data_0 (group) -> Name = "Water"
    |   |   +-- Raw_data (dataset)
    |   |   +-- PSD (dataset)
    |   |   +-- Frequency (dataset)
    |   |   +-- Treat_0 (group) -> Name = "Treat_5GHz"
    |   |   |   +-- Shift (dataset)
    |   |   |   +-- Shift_std (dataset)
    |   |   |   +-- Linewidth (dataset)
    |   |   |   +-- Linewidth_std (dataset)
    |   |   +-- Treat_1 (group) -> Name = "Treat_10GHz"
    |   |   |   +-- Shift (dataset)
    |   |   |   +-- Shift_std (dataset)
    |   |   |   +-- Linewidth (dataset)
    |   |   |   +-- Linewidth_std (dataset)
\end{verbatim}
Note that we have here added arrows and an example of the value of the "Name" attributes.
In this case, all the steps of the treatment around 5GHz are stored in the "Treat\_0" group and the ones around 10GHz in the "Treat\_1" group. The nomenclature of the attribute(s) used to store the parameters of the treatment is not specified.