In this fifth example, we are in the situation where a time-independent spectrometer has been used to acquire multiple measures. In this case, the hierrarchy of the file can be used to reduce the number of datasets, by considering that all the PSD share the same frequency axis.

The following structure represents the base structure of the file:
\begin{verbatim}
    file.h5
    +-- Data (group) -> Name = "Measure"
    |   +-- Frequency (dataset)
    |   +-- Data_0 (group) -> Name = "Sample_1"
    |   |   +-- Raw_data (dataset)
    |   |   +-- PSD (dataset)
    |   +-- Data_1 (group) -> Name = "Sample_2"
    |   |   +-- Raw_data (dataset)
    |   |   +-- PSD (dataset)
\end{verbatim}
Note that we have here added arrows and an example of the value of the "Name" attributes.