In this eleventh example, we are in the situation where multiple mappings of different dimensions have been obtained with different spectrometers that all return an array of points for all the points mapped. In this case, the hierrarchy of the file cannot be used to reduce the number of datasets, and each group will need its own abscissa and frequency datasets.

The following structure represents the base structure of the file:
\begin{verbatim}
    file.h5
    +-- Data (group) -> Name = "Measure"
    |   +-- Data_0 (group) -> Name = "VIPA"
    |   |   +-- Raw_data (dataset) [X, Y, M]
    |   |   +-- PSD (dataset) [X, Y, N]
    |   |   +-- Abscissa_0 (dataset) [X] -> Name = "x (mm)"
    |   |   +-- Abscissa_1 (dataset) [Y] -> Name = "y (mm)"
    |   |   +-- Frequency (dataset) [N]
    |   |   +-- Treat_1 (group) -> Name = "Treat"
    |   |   |   +-- Shift (dataset) [X, Y]
    |   |   |   +-- Shift_std (dataset) [X, Y]
    |   |   |   +-- Linewidth (dataset) [X, Y]
    |   |   |   +-- Linewidth_std (dataset) [X, Y]
    |   +-- Data_1 (group) -> Name = "TFP"
    |   |   +-- Raw_data (dataset) [X, Y, M]
    |   |   +-- PSD (dataset) [X, Y, N]
    |   |   +-- Abscissa_0 (dataset) [X] -> Name = "x (mm)"
    |   |   +-- Abscissa_1 (dataset) [Y] -> Name = "y (mm)"
    |   |   +-- Frequency (dataset) [N]
    |   |   +-- Treat_1 (group) -> Name = "Treat"
    |   |   |   +-- Shift (dataset) [X, Y]
    |   |   |   +-- Shift_std (dataset) [X, Y]
    |   |   |   +-- Linewidth (dataset) [X, Y]
    |   |   |   +-- Linewidth_std (dataset) [X, Y]
\end{verbatim}
Note that we have here added arrows and an example of the value of the "Name" attributes.