In this eighth example, we want to store a mapping of a sample. This mapping has been obtained with a spectrometer that returns an array of points for all the points mapped. To clarify this example, we will indicate the dimension of each dataset here between brackets.

The following structure represents the base structure of the file:
\begin{verbatim}
    file.h5
    +-- Data (group) -> Name = "Measure"
    |   +-- Data_0 (group) -> Name = "Sample"
    |   |   +-- Raw_data (dataset) [X, Y, M]
    |   |   +-- PSD (dataset) [X, Y, N]
    |   |   +-- Frequency (dataset) [N]
    |   |   +-- Abscissa_0 (dataset) [X] -> Name = "x (mm)"
    |   |   +-- Abscissa_1 (dataset) [Y] -> Name = "y (mm)"
    |   |   +-- Treat_1 (group) -> Name = "Treat"
    |   |   |   +-- Shift (dataset) [X, Y]
    |   |   |   +-- Shift_err (dataset) [X, Y]
    |   |   |   +-- Linewidth (dataset) [X, Y]
    |   |   |   +-- Linewidth_err (dataset) [X, Y]
\end{verbatim}
Note that we have here added arrows and an example of the value of the "Name" attributes.