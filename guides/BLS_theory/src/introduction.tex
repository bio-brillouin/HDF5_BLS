Brillouin Light Scattering (BLS) describes the scattering of photons by acoustic phonons. Before entering in the details of the theory, we here give an overview of the concepts at the core of BLS.

\subsection{A phonon discussion}

Phonons are part of the big quantum world where physician have come to describe matter as quantized. Phonons are quantized modes of vibration of a system. They are the quantum description of an elementary vibrational mode (a vibration at a specific frequency) of a system. Practically, phonons are quasi-particles: they don't exist persay, but are very useful to describe the vibrations of a system, particularly in the context of energy exchanges such as in BLS. 

In BLS, photons - the quanta of light and electromagnetic radiation - interact with a specific type of phonon: acoustic phonons. Acoustic phonons are used to describe acoustic waves, meaning waves implying a coherent movement of molecules. Another type of phonon is optical phonons, which are used to describe movement of molecules at much higher frequencies, usually happening between atoms of the same molecule. This type of scattering is called "Raman scattering". The fundation of BLS and Raman scattering are the same: the scattering of photons by phonons, their applications are however different:
\begin{itemize}
    \item BLS gives information about acoustic waves, meaning we can extract from BLS experiments, information about the acoustic properties of a system.
    \item Raman gives information about the vibrational modes inside a molecule, which can be used to extract information about the molecular structure, practically its chemical properties.
\end{itemize}